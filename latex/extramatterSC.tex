\documentclass[12pt]{article}
\usepackage{epsf}
\usepackage{amsmath,amssymb}
\usepackage{cancel}

\usepackage[dvips]{graphicx}

\usepackage{comment}

\setlength{\textwidth}{16.5cm}
\setlength{\textheight}{21.5cm}
\setlength{\oddsidemargin}{0cm}
\setlength{\evensidemargin}{0cm}
\setlength{\topmargin}{0cm}
\setlength{\footskip}{1cm}

\renewcommand{\topfraction}{1.0}
\renewcommand{\bottomfraction}{1.0}

\begin{document}

\newcommand{\mathhyphen}{\mathchar"712D}
\newcommand{\tr}{{\rm tr}}
\newcommand{\eps}[1]{\varepsilon_{#1}}
\newcommand{\epsD}{\bar{\varepsilon}_{\rm D}}
\newcommand{\msbar}{\overline{\rm MS}}
\renewcommand{\det}{{\rm Det}}

\newcommand{\fnc}{{\cal G}}

\newcommand{\rem}[1]{{$\spadesuit$\bf #1$\spadesuit$}}

\renewcommand{\theequation}{\thesection.\arabic{equation}}

\renewcommand{\thefootnote}{\fnsymbol{footnote}}
\setcounter{footnote}{0}

\begin{titlepage}

\begin{center}

\hfill UT-18-??\\
%\hfill January, 2018\\
\hfill \today

\vskip .75in

{\Large\bf
  Decay Rate of the Electroweak Vacuum\\[1mm]
  in the Standard Model and Beyond\\
}

\vskip .5in

{\large
  So Chigusa$^{\rm (a)}$, 
  Takeo Moroi$^{\rm (a)}$ and
  Yutaro Shoji$^{\rm (b)}$
}

\vskip 0.25in

$^{\rm (a)}${\em 
  Department of Physics, University of Tokyo, Tokyo 113-0033, Japan}

\vskip 0.1in
$^{\rm (b)}${\em 
Institute for Cosmic Ray Research, The University of Tokyo, 
Kashiwa 277-8582, Japan}


\end{center}

\vskip .5in

\begin{abstract}

  The decay rate of the electroweak vacuum is studied in the standard
  model as well as in various models beyond the standard model. We use
  a state-of-the-art technique to calculate the decay rate of a false
  vacuum in gauge theory.  We present a detailed formula to calculate
  radiative corrections to the decay rate, which can be applied to any
  model that exhibits classical conformality at a high energy scale.

  \rem{More later}

%  With
%  expanded analytic formulae for one-loop corrections, one can evaluate a
%  decay rate at the one-loop level almost instantaneously.
  
\end{abstract}

\rem{
$X$ for mode functions should be replaced by something else.
$J_G$ may be confusing because we use $J$ for angular momentum.  Maybe
$J_G\rightarrow \mathcal J_G$???
Don't use the word ``cut-off'' for the $R$-integration, because it is
misleading.
}


\end{titlepage}

\setcounter{page}{1}
\renewcommand{\thefootnote}{\#\arabic{footnote}}
\setcounter{footnote}{0}

%%%%%%%%%%%%%%%%%%%%
%Introduction
\section{Introduction}
\label{sec_intro}
\setcounter{equation}{0}

In the standard model (SM) of particle physics, it has been known that
the Higgs quartic coupling may become negative at a high scale through
quantum corrections, so that the Higgs potential develops a deeper
vacuum.  The detailed shape of the Higgs potential depends on the
Higgs and top masses; with the recently observed Higgs mass of $\sim
125\ {\rm GeV}$, it has been known that the electroweak (EW) vacuum is
not absolutely stable if the SM is valid up to $\sim 10^{10}\ {\rm
  GeV}$ or higher.\footnote
%%
{For the absolute stability of the EW vacuum in the standard model,
  see
  \cite{Degrassi:2012ry,Alekhin:2012py,Cabibbo:1979ay,Hung:1979dn}.
  \rem{Maybe, need more!!!!}
}
%%
In such a case, the EW vacuum can decay into the deeper vacuum through
tunneling in quantum field theory.  The lifetime of the EW vacuum has
been one of the important topics in particle phenomenology and cosmology
\cite{???}.  In addition, the decay rate is highly sensitive to new
particles that couple to the Higgs boson because of additional quantum
corrections to the Higgs effective potential \cite{????}. It allows us
to constrain models beyond the SM even when new particles are much
heavier than the EW scale.\footnote
%%
{\rem{Comment on the Higgs dynamics during inflation????}}
%%

The decay rate of the EW vacuum has been discussed for a long time.
The calculation of the decay rate at the one-loop level first appeared
in \cite{Isidori:2001bm}, and was also discussed in other literature
\cite{????, Andreassen:2017rzq, Chigusa:2017dux}.\footnote
%%
{For earlier works on the decay rate of the EW vacuum, see
  \cite{????}.}
%%
However, there are subtleties in the treatment of zero modes related
to the gauge symmetry breaking, which make it difficult to perform a
precise and reliable calculation of the decay rate.  Recently, a
prescription for the treatments of the gauge zero modes has been
clarified \cite{Endo:2017gal,Endo:2017tsz}, based on which complete
calculations for the SM case were carried out
\cite{Andreassen:2017rzq,Chigusa:2017dux}.

The lifetime of a vacuum can be evaluated through a rate of bubble
nucleation in unit volume and unit time as formulated in
\cite{Coleman:1977py,Callan:1977pt}.  The rate is expressed in the
form of
\begin{align}
 \gamma = \mathcal Ae^{-\mathcal B},
\end{align}
where $\mathcal B$ is the action of a so-called bounce solution, and
prefactor $\mathcal A$ is essentially quantum corrections to $\mathcal
B$, having mass dimension 4.  Bounce solution is an $O(4)$ symmetric
solution of the Euclidean equations of motion, connecting the two
vacua. In the SM, there are infinite number of bounce solutions owing
to (i) the classical scale invariance at a high energy scale, (ii) the
global symmetries corresponding to $SU(2)_L\times U(1)_Y/U(1)_{\rm
  EM}$, as well as (iii) the translational invariance.  It has been
known that (iii) represents the infinite volume of spacetime, giving
the mass dimension to $\mathcal A$ \cite{Callan:1977pt}.  On the
contrary, a proper procedure to take account of the effects of the
zero-modes related to (i) and (ii) were not well understood until
\cite{Andreassen:2017rzq,Chigusa:2017dux}.  Although the dominant
suppression of the decay rate comes from $\mathcal B$, the prefactor
$\mathcal A$ is also important. This is because of large quantum
corrections from the top quarks and the gauge bosons. Thus, it is
essential to calculate both $\mathcal A$ and $\mathcal B$ to determine
the decay rate precisely.  In this paper, we calculate $\mathcal A$
properly taking into account the effects of zero-modes, which gives a
reliable estimation of the decay rate.

We first present a detailed formulation of the calculation of the
decay rate at the one-loop level. Each of the one-loop corrections is
given in the form of a ratio of functional determinants.  We obtain a
complete set of analytic formulae that can be used for any models that
exhibit classical conformality at a high energy scale like the SM.
Then, as one of the important applications, we calculate the decay
rate of the EW vacuum in the SM. We find that the decay rate of the EW
vacuum is much longer than the age of the universe. We also find that
one-loop corrections from the top quark and the gauge bosons are very
large although there is an accidental cancellation. It shows the
importance of $\mathcal A$ for the evaluation of a decay rate.  We
also evaluate the decay rates of the EW vacuum for models with extra
fermions which couples to the Higgs field.  In such models, the EW
vacuum tends to be destabilized compared with that of the SM
\cite{Kobakhidze:2013pya, Khan:2012zw, Dev:2013ff, Rodejohann:2012px,
  Xiao:2014kba} since the quartic coupling of the Higgs field is
strongly driven into a negative value. We consider three cases where
we have, in addition to the SM particles, (i) vector-like fermions
having the same SM charges as left quark and down quark, $(Q_{\rm
  ex},\bar{Q}_{\rm ex}, D_{\rm ex}, \bar{D}_{\rm ex})$, (ii)
vector-like fermions having the same SM charges as left lepton and
electron, $(L_{\rm ex},\bar{L}_{\rm ex}, E_{\rm ex}, \bar{E}_{\rm
  ex})$, and (iii) a right-handed neutrino, $(N_R)$. We give
constraints on their couplings and masses, requiring that the lifetime
of the EW vacuum is longer than the age of the universe.

\rem{TM: I will check the following paragraph later}

This paper is organized as follows. In section \ref{sec_formulation},
we outline the formulation for the decay rate at the one-loop
level. The detail of the calculation is given in Appendices
\ref{apx_determinantJ} $-$ \ref{apx_divpart}.  We provide analytic
formulae for each particle that couples to the Higgs boson and discuss
resummation of logarithmic terms using the renormalization group
(RG). In section \ref{sec_SM}, we evaluate the vacuum decay rate in
the SM. All the technical details are covered in this section. In
section \ref{sec_extra}, we analyze decay rates in models with extra
fermions. Finally, we conclude in section \ref{sec_conclusion}.

%%%%%%%%%%%%%%%%%%%%
%Formulation
\section{Formulation}
\label{sec_formulation}
\setcounter{equation}{0}

We first discuss how we calculate the decay rate of the EW vacuum.  In
the SM, the EW vacuum becomes unstable due to the RG running of the
quartic coupling constant of the Higgs boson, which makes the quartic
coupling constant negative at a high scale.  In the SM, the
instability occurs when the Higgs amplitude becomes much larger than
the EW scale.  Then, the typical size of the field value for the
bounce configuration is much larger than the EW scale.  In such a
case, we can safely neglect the quadratic term in the Higgs potential
because the mass-squared parameter for the quartic term is around the
EW scale.

In this section, in order to discuss how the decay rate of the false
vacuum in such a situation can be evaluated, we use a toy model with
$U(1)$ gauge symmetry to derive relevant formulas.  The calculation of
the decay rate of the EW vacuum is almost parallel to that in the case
with $U(1)$ gauge symmetry; the application of the formulas to the
case of the SM will be explained in the next section.

\subsection{Set up}

\rem{Comment on the Planck suppressed operator.}

Let us first summarize the set up of our analysis.  We study the decay
rate of a false vacuum whose instability is due to an RG running of
the quartic coupling constant of a scalar field $\Phi$.  We assume that
$\Phi$ is charged under the $U(1)$ gauge symmetry (with charge $+1$);
the kinetic term includes 
\begin{align}
  {\cal L}_{\rm kin} \ni
  \left[
    \left(
      \partial_\mu - i g A_\mu 
    \right) \Phi
  \right]^\dagger
  \left(
    \partial_\mu - i g A_\mu 
  \right) \Phi,
\end{align}
where $A_\mu$ is the gauge field and $g$ is the gauge coupling
constant, while the scalar potential is approximated as
\begin{align}
  V(\Phi) = \lambda(\Phi^\dagger\Phi)^2.
  \label{V(Phi)}
\end{align}
$\lambda$ depends on the renormalization scale $\mu$ and is assumed to
become negative at a high scale due to the RG effect.  Notice that we
neglect the mass-squared term in the scalar potential, assuming that
$\lambda$ becomes negative at a scale much higher than the mass
parameter in the Lagrangian.  With the set up mentioned above, it is
notable that the scalar potential has scale-invariance at the classical
level.  In the application to the case of the SM, $\Phi$ corresponds to
the Higgs doublet and $\lambda$ corresponds to the Higgs quartic
coupling constant.

Hereafter, we perform a detailed study of the effects of the fields
coupled to the scalar field $\Phi$ on the decay rate of the false
vacuum.  We consider the Lagrangian which contains the following
interaction terms:
\begin{align}
  \mathcal L_{\rm int} \ni
  \kappa\sigma^2|\Phi|^2
  + ( y \Phi \bar{\psi}_L \psi_R + {\rm h.c.} )
  + V(\Phi),
\end{align}
where $\sigma$ is a real scalar field, and $\psi_L$ and $\psi_R$ are
chiral fermions (with relevant $U(1)$ charges).\footnote
%%
{We assume that there exist other chiral fermions which cancel out
  the gauge anomaly.}
%%
We take $y$ real, and $\kappa>0$.  We neglect dimensionful parameters
which are assumed to be much smaller than the typical scale of the
bounce.  In addition, in order to take into account the effects of
gauge-boson loops, gauge fixing is necessary.  Following
\cite{Kusenko, EMNS2}, we take the gauge fixing function of the
following form:
\begin{align}
 \mathcal F = \partial_\mu A_\mu.
\end{align}
Then, the gauge fixing term and the Lagrangian of the ghosts (denoted
as $\bar{c}$ and $c$) are given by
\begin{align}
  \mathcal L_{\rm GF} & = \frac{1}{2\xi}\mathcal F^2, \\
  \mathcal L_{\rm FP} & = - \bar{c} \partial_\mu \partial_\mu c,
  \label{L_FP}
\end{align}
where $\xi$ is the gauge-fixing parameter.\footnote
%%
{In the non-Abelian case, one of $\partial_\mu$ in Eq.\ \eqref{L_FP}
  is replaced by the covariant derivative. The interaction of the
  ghosts with the gauge field does not affect the following
  discussion.}
%%

With our choice of the gauge fixing function, a global $U(1)$ symmetry
remains in the Lagrangian.  As a result, the bounce solution can be
written in the following form:
\begin{align}
  \Phi|_{\rm bounce} = \frac{1}{\sqrt{2}} e^{i\theta}
  \bar\phi(r),
\end{align}
where $\theta$ is a real parameter.  The function $\bar{\phi}$ obeys
\begin{align}
  \partial_r^2\bar\phi(r)
  +\frac{3}{r}\partial_r\bar\phi(r)
  -\lambda\bar\phi^3(r) = 0,
  \label{eq_Bounce}
\end{align}
with boundary conditions $\partial_r\bar\phi(0) = 0$ and
$\bar\phi(\infty) = 0$.  For a negative $\lambda$, we have a series of
Fubini-Lipatov instanton solutions \cite{Fubini:1976jm,Lipatov:1976ny}:
\begin{align}
  \bar\phi(r) = \frac{\bar\phi_C}{1+R^{-2} r^2},
  \label{BounceSolution}
\end{align}
which is parameterized by $\bar\phi_C$ (i.e., the field value at the
center of the bounce) and
\begin{align}
  R \equiv \sqrt{\frac{8}{|\lambda|}} \phi_C^{-1}.
\end{align}
The bounce action is given by
\begin{align}
 \mathcal B = \frac{8\pi^2}{3|\lambda|}.
\end{align}
Notice that the bounce action is independent of $\bar\phi_C$ owing to
the scale invariance.

Once the bounce solution is obtained, we may integrate over the
fluctuation around it.  Around the bounce configuration with
$\theta=0$, we may expand
\begin{align}
  \Phi = \frac{1}{\sqrt{2}} \left( 
    \bar\phi + h + i \varphi
  \right),
\end{align}
where $h$ and $\varphi$ are physical Higgs mode and Nambu-Goldstone
(NG) mode, respectively.  At the one-loop level, the prefactor can be
decomposed as a product of the contributions of individual fields:
\begin{align}
  \mathcal A e^{-\mathcal B} = \frac{\mathcal A^{(h)}}{\mathcal V_{\rm 4D}}
  \mathcal A^{(\sigma)}\mathcal A^{(\psi)}\mathcal A^{(A_\mu,\varphi)}
  \mathcal A^{(c,\bar{c})}e^{-\mathcal B},
  \label{eq_AeB}
\end{align}
where $\mathcal V_{\rm 4D}$ is the volume of spacetime, $A^{(X)}$ is
the contribution from particle $X$.  Here, $A_\mu$ is the gauge
field. Each of the factors has a form of
\begin{align}
 \mathcal A^{(X)} = \left(
  \frac{\det\mathcal M^{(X)}}{\det \widehat{\mathcal M}^{(X)}}
 \right)^{w^{(X)}},
\end{align}
where $\mathcal M^{(X)}$ and $\widehat{\mathcal M}^{(X)}$ are the
fluctuation operators around the bounce solution and around the false
vacuum, respectively.  Here, $w^{(X)} = 1$ for Dirac fermions and
Faddeev-Popov ghosts, and $w^{(X)} = -1/2$ for the other bosonic
fields.  The fluctuation operator is defined as second derivatives of
the action:
\begin{align}
 \mathcal M^{(X)} \delta^4(x-y) = \left\langle
 \frac{\delta^2S(X)}{\delta X(x)\delta X(y)}
 \right\rangle,
\end{align}
where the brackets indicate the expectation value with the bounce
solution. Since the Faddeev-Popov ghosts do not couple to the Higgs
boson with the present choice of the gauge fixing function, ${\mathcal
  M}^{(c,\bar{c})}= \widehat{\mathcal M}^{(c,\bar{c})}$, and
\begin{align}
  \mathcal A^{(c,\bar{c})}=1.
\end{align}

\subsection{Functional determinant}

\rem{Just for the $U(1)$ case; superscript ``a'' is not necessary!!!!!}

For the evaluation of the functional determinants, we first decompose
the fluctuations into partial waves, making use of the $O(4)$ symmetry
of the bounce;
\begin{align}
 \frac{\det\mathcal M^{(X)}}{\det \widehat{\mathcal M}^{(X)}}
 = \prod_{J=0}^{\infty}
 \left(
  \frac{\det \mathcal M^{(X)}_J}{\det \widehat{\mathcal M}^{(X)}_J}
 \right)^{n^{(X)}_J},
\end{align}
where $\mathcal M^{(X)}_J$ is the fluctuation operator for each
partial wave labeled by non-negative half-integer $J =
0,1/2,1,\cdots$. Here, $n^{(\psi)}_J = (2J+1)(2J+2)$ for fermions, and
$n^{(X)}_J = (2J+1)^2$ for the others.  The explicit form of the
fluctuation operators are given in Appendix
\ref{apx_determinantJ}. From a theorem \cite{Gelfand:1959nq,
  Dashen:1974ci, Kirsten:2003py, Kirsten:2004qv, Endo:2017tsz}, we
have
\begin{align}
 \frac{\det \mathcal M^{(X)}_J}{\det \widehat{\mathcal M}^{(X)}_J} =
 \left(
  \lim_{r\to\infty}\frac{\det\Psi}{\det\hat\Psi}
 \right)
 \left(
  \lim_{r\to0}\frac{\det\Psi}{\det\hat\Psi}
 \right)^{-1},
 \label{eq_theorem}
\end{align}
where $\Psi = (\Psi_1,\Psi_2,\cdots)$ and $\hat\Psi =
(\hat\Psi_1,\hat\Psi_2,\cdots)$ are sets of independent solutions of
\begin{align}
  \mathcal M^{(X)}_J\Psi_i = &\, 0,\\
  \widehat{\mathcal M}^{(X)}_J\hat\Psi_i = &\, 0,
\end{align}
and are regular at $r = 0$. Notice that, when $\mathcal M^{(X)}_J$ is
$n\times n$ object, there are $n$ independent solutions that are
regular at $r = 0$. Since $\Psi_i$ and $\hat\Psi_i$ obey the same
linear differential equations at $r\to\infty$, the ratio of the two
determinant converges for each $J$.

With a Fubini-Lipatov instanton, we can calculate the ratio
analytically; the details of the calculation are given in Appendix
\ref{apx_determinantJ}.  The ratios are given by
\begin{align}
  \frac{\det \mathcal M^{(h)}_J}{\det \widehat{\mathcal M}^{(h)}_J}
  & = \frac{2J(2J-1)}{(2J+3)(2J+2)},
  \label{detM^h}
  \\
  \frac{\det \mathcal M^{(\sigma)}_J}{\det \widehat{\mathcal M}^{(\sigma)}_J}
  & = \frac{\Gamma(2J+1)\Gamma(2J+2)}{\Gamma(2J+1-z_\kappa)\Gamma(2J+2+z_\kappa)},
  \\
  \frac{\det \mathcal M^{(\psi)}_J}{\det \widehat{\mathcal
      M}^{(\psi)}_J}
  & =
  \left[
    \frac{[\Gamma(2J+2)]^2}{\Gamma\left(2J+2-z_y\right)\Gamma\left(2J+2+z_y\right)}
  \right]^2,
  \\
  \frac{\det \mathcal M^{(A_\mu,\varphi)}_J}{\det \widehat{\mathcal M}^{(A_\mu,\varphi)}_J}
  & = \frac{J}{J+1}
  \left[
    \frac{\Gamma(2J+1)\Gamma(2J+2)}{\Gamma(2J+1-z_g)\Gamma(2J+2+z_g)}
  \right]^3,
  \label{detM^gauge}
\end{align}
with
\begin{align}
  z_\kappa = & -\frac{1}{2}
  \left(
    1-\sqrt{1-8\frac{\kappa}{|\lambda|}}
  \right),\\
  ~z_g = & -\frac{1}{2}
  \left(
    1-\sqrt{1-8\frac{g^2}{|\lambda|}}
  \right),\\
  ~z_y = &
  i\frac{y}{\sqrt{\lambda}}.
\end{align}
Here, $g$ is the gauge coupling constant, and $\Gamma(z)$ is the
gamma function.

\subsection{Zero mode}

In the calculation of the decay rate of the EW vacuum with the present
set up, there show up zero modes in association with dilatation,
translation, and global transformation of the bounce solution.
Consequently, $\mathcal M_0^{(h)},\mathcal M_{1/2}^{(h)}$ and
$\mathcal M_0^{(A_\mu,\varphi)}$ have zero eigenvalues, and their
determinants vanish as shown in eqs.\ \eqref{detM^h} and
\eqref{detM^gauge}.  A naive inclusion of such results gives a
divergent behavior of the decay rate, and hence careful treatments of
the zero modes are necessary.

When there is a zero mode, the saddle point approximation for the path
integral breaks down and we need to go back to the original path
integral.  Let us consider the path integral for a bosonic field.  We
first decompose the fluctuation of the bosonic field (denoted as $X$)
by using the eigenfunctions of the fluctuation operator $\mathcal
M^{(X)}$:
\begin{align}
  X = \sum_ic_i \fnc_i,
\end{align}
where $c_i$ is the expansion coefficient while $X_i$ obeys
\begin{align}
  \mathcal M^{(X)} \fnc_i = \omega_i \fnc_i,
\end{align}
with $\omega_i$ being the eigenvalue.  We denote the zero mode as
$\fnc_0$ (and hence $\lambda_0=0$).  In addition, we define 
\begin{align}
  \langle \fnc_i| \fnc_j \rangle = \frac{\delta_{ij}}{\mathcal N_i^2}.
  \label{eq_normalization}
\end{align}
where the inner product is given by
\begin{align}
 \langle \fnc_i| \fnc_j\rangle = \int d^4x \fnc_i^\dagger(x) \fnc_j(x).
\end{align}
We leave $\mathcal N_i$'s unspecified since the final result is
independent of them.  

The path integral for a bosonic field $X$ is then defined as
\begin{align}
 \int \mathcal DXe^{-S^{(X)}[X]}
 = \int \prod_i\frac{dc_i}{\sqrt{2\pi}\mathcal N_i}e^{-S^{(X)}[X]},
\end{align}
where $S^{(X)}[X]$ denotes the $X$-dependent terms in the action.  (We
take $S^{(X)}[X=0]=0$.)  Applying the saddle point approximation for
the modes other than the zero mode, the path integral is evaluated as
\begin{align}
  \int \prod_i\frac{dc_i}{\sqrt{2\pi}\mathcal N_i}e^{-S^{(X)}[X]}
  = \int \frac{dc_0}{\sqrt{2\pi}\mathcal N_0}
  \prod_{i\neq0}\frac{1}{\sqrt{\omega_i}}.
\end{align}
Notice that $\mathcal M^{(X)}$, $X_i$ and $\omega_i$ may depend on
$c_0$ \cite{Weinberg???}.

We are interested in the case where there exists a symmetry and the
Lagrangian is invariant under a transformation (parameterized by $z$).
Then, the zero mode is in association with such a symmetry.  In such a
case, with the transformation of the bounce solution, $X$ varies as
\begin{align}
 X \to X + z \tilde{\fnc}_0+\mathcal O(z^2),
\end{align}
where the function $\tilde{\fnc}_0$ is proportional to $\fnc_0$.  The
integration over $c_0$ can be understood as the integration over the
collective coordinate $z$:
\begin{align}
  \frac{dc_0}{\sqrt{2\pi}\mathcal N_0}
%   = 
%   dz
%   \sqrt{
%     \frac{\langle\tilde{\fnc}_0|\tilde{\fnc}_0\rangle}
%     {\langle \fnc_0|\fnc_0\rangle}
%   }
%   \frac{1}{\sqrt{2\pi}\mathcal N_0}
  \rightarrow
  dz\sqrt{\frac{\langle\tilde{\fnc}_0|\tilde{\fnc}_0\rangle}{2\pi}}.
  \label{eq_dzIntegral}
\end{align}

Next, we discuss how we evaluate
\begin{align*}
  \sqrt{\frac{\langle\tilde{\fnc}_0|\tilde{\fnc}_0\rangle}{2\pi}}
  \prod_{i\neq0}\frac{1}{\sqrt{\omega_i}}.
\end{align*}
To omit the zero eigenvalue from the functional determinant, we
introduce a regulator to the fluctuation operator:
%% 
\begin{align}
  \mathcal M^{(X)}\to\mathcal M^{(X)} + \nu \rho(x),
\end{align}
where $\nu$ is a small positive number, and $\rho(x)$ is an arbitrary
function that satisfies
\begin{align}
  \langle \tilde{\fnc}_0|\rho|\tilde{\fnc}_0\rangle = 2\pi.
  \label{eq_normalization}
\end{align}
Then, the path integral for the regularized fluctuation operator is
given by
\begin{align}
  \int \prod_{i}\frac{dc_i}{\sqrt{2\pi}\mathcal N_i}
  e^{-\frac{1}{2}c_jc_k\langle X_j|({\mathcal M}^{(X)}+\nu \rho)|X_k\rangle}
  = \sqrt{\frac{\langle\tilde{\fnc}_0|\tilde{\fnc}_0\rangle}{2\pi}}
  \frac{1}{\sqrt{\nu+\mathcal O(\nu^2)}}
  \prod_{i\neq 0}\frac{1}{\sqrt{\omega_i+\mathcal O(\nu)}},
\end{align}
resulting in
\begin{align}
  \sqrt{\frac{\langle\tilde{\fnc}_0|\tilde{\fnc}_0\rangle}{2\pi}}
  \prod_{i\neq0}\frac{1}{\sqrt{\omega_i}}
  = \lim_{\nu\to0}\sqrt{\nu}\int \prod_{i}\frac{dc_i}{\sqrt{2\pi}\mathcal N_i}
  e^{-\frac{1}{2}c_jc_k\langle X_j|({\mathcal M}^{(X)}+\nu \rho)|X_k\rangle}.
\end{align}
The integration in the above expression is nothing but $\det ({\mathcal
M}^{(X)}+\nu \rho)$, and can be evaluated with the use of
eq.~\eqref{eq_theorem}.  For example, when $\mathcal M_J^{(X)}$ is a
$1\times1$ object, we have
\begin{align}
 \left(
  \frac{\det\mathcal M^{(X)}_J}{\det\widehat{\mathcal M}^{(X)}_J}
 \right)^{-1/2}
 \to
% \int dz
% \lim_{\nu\to0}\sqrt{\nu}
 \left(
   \lim_{r\to\infty}\frac{\check\Psi(r)}{\hat\Psi(r)}
 \right)^{-1/2},
 \label{eq_detPrime}
\end{align}
where
\begin{align}
 \mathcal M_J^{(X)}\Psi(r)               & = 0,                  \\
 \widehat{\mathcal M}_J^{(X)}\hat\Psi(r) & = 0,                  \\
 \mathcal M_J^{(X)}\check\Psi(r)         & = 
% -\nu \rho(r)\Psi(r)
 - \rho(r)\Psi(r)
 \label{eq_detPrimeFunc},
\end{align}
with $\lim_{r\to0}\Psi(r)/\hat\Psi(r) = 1$.  Here, we choose $\rho$ to
be a function of $r$.
%
% $\rho(r)$ is the radial
%part of $\rho(x)$.  Since this is a linear differential equation, we can
%omit $\nu$ from eqs.~\eqref{eq_detPrime} and \eqref{eq_detPrimeFunc}.

From Appendix \ref{apx_zeromode}, we have the following replacements
to take care of the zero modes:
\begin{align}
  \left|
    \frac{\det \mathcal M^{(h)}_0}{\det \widehat{\mathcal M}^{(h)}_0}
  \right|^{-1/2}
  & \to\int d \ln R
  \sqrt{\frac{16\pi}{|\lambda|}},
  %frac{1}{\bar\phi_C},
  \label{eq_dilatationalZeromode}                                           \\
  \left(
    \frac{\det \mathcal M^{(h)}_{1/2}}{\det \widehat{\mathcal M}^{(h)}_{1/2}}
  \right)^{-2}
  & \to \left(
    \frac{32\pi}{|\lambda|}\right)^2\left(\frac{|\lambda|}{8}\bar\phi_C^2
  \right)^2
  {\mathcal V}_{\rm 4D},
  \\
  \left(
    \frac{\det \mathcal M^{(A^a,\varphi^a)}_0}
    {\det \widehat{\mathcal M}^{(A^a,\varphi^a)}_0}
  \right)^{-1/2}
  & \to \int d\theta^aJ_G\sqrt{\frac{16\pi}{|\lambda|}},
\end{align}
with
\begin{align}
  J_G = \frac{\sqrt{2|\langle T^a\Phi\rangle|^2}}{|\bar\phi|}.
\end{align}


\rem{TM: $J_G$ and gauge volume, how shall we do???}


\rem{I should rather say something different; as Matt, we can say we
  have confirmed $\det\mathcal M_0^{(h)}<0$.
Here, we take an absolute value for $\det\mathcal M_0^{(h)}$ since it
has a negative eigenvalue.
}

\subsection{Renormalization}

After taking the product over $J$, we have a UV divergence and thus we
need to renormalize the result.  In this paper, we use the
$\msbar$-scheme for the renormalization. Notice that the counter terms
can subtract the divergence even when the background field is position
dependent since the counter terms are apparently momentum independent.

\rem{TM: above statement???}

Since the dimensional regularization cannot be directly used in this
evaluation, we first regularize the result as
\begin{align}
  \left[
    \ln\mathcal A^{(X)}
  \right]_{\eps{X}}
  \equiv w^{(X)}\sum_{J=0}^{\infty}
  \frac{n^{(X)}_J}{(1+\eps{X})^{2J}}\ln
  \left(
    \frac{\det \mathcal M^{(X)}_J}{\det \widehat{\mathcal M}^{(X)}_J}
  \right),
\end{align}
where $\eps{X}$ is a positive number which will be taken to be zero at
the end of the calculation. Since the divergence is at most a power of
$J$, the sum converges.  In Appendix \ref{apx_infiniteSum}, we
calculate the sum analytically, and obtain
% get the
% following results after summing over $J$:
\begin{align}
  \left[
    \ln\mathcal A^{(\sigma)}
  \right]_{\eps{\sigma}}
  = &\, -\frac{1}{2}\sum_{J=0}^{\infty}\frac{(2J+1)^2}{(1+\eps{\sigma})^{2J}}
  \ln\frac{\Gamma(2J+1)\Gamma(2J+2)}{\Gamma(2J+1-z_\kappa)\Gamma(2J+2+z_\kappa)}
  \nonumber \\
  = &\, -\frac{1}{2}
  \left[
    \frac{2\kappa}{|\lambda|}
    \left(
      \frac{1}{\eps{\sigma}^2}+\frac{2}{\eps{\sigma}}
      +\frac{\kappa}{3|\lambda|}\ln\eps{\sigma}
    \right)+\mathcal S_B(z_\kappa)
  \right]+\mathcal O(\eps{\sigma}),
%   \\
%   \left[
%     \ln\mathcal A'^{(h)}
%   \right]_{\eps{h}}
%   & = -\frac{1}{2}
%   \left[
%     \left(
%       \ln\frac{|\lambda|}{16\pi}
%     \right)+4
%     \left(
%       \ln\frac{|\lambda|}{32\pi}
%     \right)+\sum_{J=1}^{\infty}\frac{(2J+1)^2}{(1+\eps{h})^{2J}}
%     \ln\frac{2J(2J-1)}{(2J+3)(2J+2)}
%   \right]
%   \nonumber \\
%   & = -\frac{1}{2}
%   \left[
%     -6
%     \left(
%       \frac{1}{\eps{h}^2}+\frac{2}{\eps{h}}-\ln\eps{h}
%     \right)\right.
%   \nonumber \\
%   & \left.\hspace{7ex}+\frac{3}{2}+6\gamma_E+12\ln A_G-5
%     \ln\frac{\pi}{3}+5\ln\frac{|\lambda|}{8}
%   \right]+\mathcal O(\eps{h}),
  \\
  \left[
    \ln\mathcal A^{(\psi)}
  \right]_{\eps{\psi}}
  = &\, \sum_{J=0}^{\infty} \frac{(2J+1)(2J+2)}{(1+\eps{\psi})^{2J}}
  \ln\left[
    \frac{[\Gamma(2J+2)]^2}{\Gamma
      \left(
        2J+2-i\frac{y}{\sqrt{\lambda}}
      \right)\Gamma
      \left(
        2J+2+i\frac{y}{\sqrt{\lambda}}
      \right)}
  \right]^2
  \nonumber \\
  = &\, \frac{y^2}{|\lambda|}
  \left(
    \frac{2}{\eps{\psi}^2}+\frac{5}{\eps{\psi}}
    +\frac{1}{3}\frac{y^2+2|\lambda|}{|\lambda|}\ln\eps{\psi}
  \right)+\mathcal S_F(z_y)+\mathcal O(\eps{\psi}).
%   \\
%   \left[
%     \ln\mathcal A'^{(A_\mu^a,\varphi^a)}
%   \right]_{\eps{A}}
%   & = -\frac{1}{2}
%   \left[
%     \left(
%       \ln\frac{|\lambda|}{16\pi J_G^2}
%     \right)\right.
%   \nonumber \\
%   & \left.\hspace{7ex}+\sum_{J=1/2}^{\infty}\frac{(2J+1)^2}{(1+\eps{A})^{2J}}
%     \ln\frac{J}{J+1}
%     \left(
%       \frac{\Gamma(2J+1)\Gamma(2J+2)}{\Gamma(2J+1-z_g)\Gamma(2J+2+z_g)}
%     \right)^3\right]
%   \nonumber \\
%   & = -\frac{1}{2}
%   \left[
%     \left(
%       \frac{6g^2}{|\lambda|}-2
%     \right)
%     \left(
%       \frac{1}{\eps{A}^2}+\frac{2}{\eps{A}}
%     \right)+
%     \left(
%       \frac{2}{3}+\frac{2g^4}{|\lambda|^2}
%     \right)\ln\eps{A}\right.
%   \nonumber \\
%   & \left.\hspace{7ex}-\frac{3}{2}+\frac{2}{3}\gamma_E+4\ln A_G+
%     \ln\frac{|\lambda|}{8\pi J_G^2}\right.
%   \nonumber \\
%   & \left.\hspace{7ex}+3\mathcal S_B(z_g)+3\ln\Gamma(1-z_g)\Gamma(2+z_g)
%   \right] +\mathcal O(\eps{A}).
\end{align}
\rem{We should remove 1/2, with deviding the rest by 2.   
We can eliminate the first line of each RHS.}
In addition, define
\begin{align}
  \mathcal A^{(h)}
  & = {\mathcal V}_{\rm 4D} \int d \ln \bar\phi_C
  \left(
    \frac{|\lambda|}{8}\bar\phi_C^2
  \right)^2
  \mathcal A'^{(h)},
  \\
  \mathcal A^{(A_\mu,\varphi)}
  & = \int d\theta \mathcal A'^{(A_\mu,\varphi)}.
\end{align}
Then 
\begin{align}
  \left[
    \ln\mathcal A'^{(h)}
  \right]_{\eps{h}}
  & = -\frac{1}{2}
  \left[
    \left(
      \ln\frac{|\lambda|}{16\pi}
    \right)+4
    \left(
      \ln\frac{|\lambda|}{32\pi}
    \right)+\sum_{J=1}^{\infty}\frac{(2J+1)^2}{(1+\eps{h})^{2J}}
    \ln\frac{2J(2J-1)}{(2J+3)(2J+2)}
  \right]
  \nonumber \\
  & = -\frac{1}{2}
  \left[
    -6
    \left(
      \frac{1}{\eps{h}^2}+\frac{2}{\eps{h}}-\ln\eps{h}
    \right)\right.
  \nonumber \\
  & \left.\hspace{7ex}+\frac{3}{2}+6\gamma_E+12\ln A_G-5
    \ln\frac{\pi}{3}+5\ln\frac{|\lambda|}{8}
  \right]+\mathcal O(\eps{h}),
  \\
  \left[
    \ln\mathcal A'^{(A_\mu,\varphi)}
  \right]_{\eps{A}}
  & = -\frac{1}{2}
  \left[
    \left(
      \ln\frac{|\lambda|}{16\pi J_G^2}
    \right)\right.
  \nonumber \\
  & \left.\hspace{7ex}+\sum_{J=1/2}^{\infty}\frac{(2J+1)^2}{(1+\eps{A})^{2J}}
    \ln\frac{J}{J+1}
    \left(
      \frac{\Gamma(2J+1)\Gamma(2J+2)}{\Gamma(2J+1-z_g)\Gamma(2J+2+z_g)}
    \right)^3\right]
  \nonumber \\
  & = -\frac{1}{2}
  \left[
    \left(
      \frac{6g^2}{|\lambda|}-2
    \right)
    \left(
      \frac{1}{\eps{A}^2}+\frac{2}{\eps{A}}
    \right)+
    \left(
      \frac{2}{3}+\frac{2g^4}{|\lambda|^2}
    \right)\ln\eps{A}\right.
  \nonumber \\
  & \left.\hspace{7ex}-\frac{3}{2}+\frac{2}{3}\gamma_E+4\ln A_G+
    \ln\frac{|\lambda|}{8\pi J_G^2}\right.
  \nonumber \\
  & \left.\hspace{7ex}+3\mathcal S_B(z_g)+3\ln\Gamma(1-z_g)\Gamma(2+z_g)
  \right] +\mathcal O(\eps{A}).
\end{align}

\rem{Functions ${\cal S}_{B,F}$???.  We should remove the prefactor
  $1/2$, with deviding the rest by $2$.}

Next, we relate the above results with those based on dimensional
regularizaton in $D$ dimension which contain the regularization
parameter $\epsD$ defined as
\begin{align}
 \frac{1}{\epsD} = \frac{2}{4-D}+\ln4\pi-\gamma_E,
\end{align}
with $\gamma_E$ being the Euler's constant.  We convert the results
based on two different regularizations by calculating the following
quantity:
\begin{align}
 \left[\ln\mathcal A^{(X)}\right]_{\rm div}
  & = w^{(X)}\tr
 \left[
  \left(
   \widehat{\mathcal M}^{(X)}
  \right)^{-1}\delta\mathcal M^{(X)}
 \right]-\frac{w^{(X)}}{2}\tr
 \left[
  \left(
   \widehat{\mathcal M}^{(X)}
  \right)^{-1}\delta\mathcal M^{(X)}
  \left(
   \widehat{\mathcal M}^{(X)}
  \right)^{-1}\delta\mathcal M^{(X)}
 \right]\nonumber                         \\
  & = w^{(X)}\sum_{J=0}^{\infty}n^{(X)}_J
 \left[
  \ln\frac{\det
   \left[
    \widehat{\mathcal M}^{(X)}_J+\delta\mathcal M_J^{(X)}
   \right]}{\det \widehat{\mathcal M}^{(X)}_J}
 \right]_{\mathcal O(\delta\mathcal M_J^2)}
 \label{eq_deltaMExpansion}
\end{align}
where
\begin{align}
 \delta\mathcal M^{(X)} = \mathcal M^{(X)}-\widehat{\mathcal M}^{(X)},
\end{align}
and $\delta \mathcal M^{(X)}_J$ is defined similarly. Here,
$[\cdots]_{\mathcal O(\delta)}$ indicates the expansion up to
$\delta$. The most important point is that $\left[\ln\frac{\det\mathcal
  M^{(X)}}{\det\widehat{\mathcal M}^{(X)}}\right]_{\rm div}$ has the same
divergence as $\left[\ln\frac{\det\mathcal
  M^{(X)}}{\det\widehat{\mathcal M}^{(X)}}\right]$ has when
$\delta\mathcal M^{(X)}$ does not have a derivative operator.

The first line of eq.~\eqref{eq_deltaMExpansion} can be calculated by
directly evaluating the traces in a momentum space with using the
dimensional regularization; the result is denoted as $\left[\ln
  \mathcal A^{(X)}\right]_{{\rm div}, \epsD}$. 
%Notice that
%$\widehat{\mathcal M}^{(X)}$ is practically ${\rm
%  diag}(-\partial^2,-\partial^2,\cdots)$, and thus we have a simple
%Green's function.
On the other hand, the second line of eq.~\eqref{eq_deltaMExpansion} can
be evaluated as
\begin{align}
 \left[
  \ln\frac{\det
   \left[
    \widehat{\mathcal M}^{(X)}_J+\delta\mathcal M_J^{(X)}
   \right]}{\det \widehat{\mathcal M}^{(X)}_J}
 \right]_{\mathcal O(\delta\mathcal M_J^2)}
 = \tr
 \left[
  \hat\Psi^{-1}\Psi^{(1)}
 \right]+\tr
 \left[
  \hat\Psi^{-1}\Psi^{(2)}
 \right]-\frac{1}{2}\tr
 \left[
  \hat\Psi^{-1}\Psi^{(1)}\hat\Psi^{-1}\Psi^{(1)}
 \right],
 \label{eq_expansionEachJ}
\end{align}
where
\begin{align}
 \widehat{\mathcal M}^{(X)}_J\Psi^{(p)} = -\delta\mathcal M_J^{(X)}\Psi^{(p-1)},
\end{align}
for $p=1$ and $2$, and $\Psi^{(0)} = \hat\Psi$. Then, we calculate
\begin{align}
  \left[
    \ln\mathcal A^{(X)}
  \right]_{{\rm div}, \eps{X}}
  \equiv w^{(X)}\sum_J\frac{n^{(X)}_J}{(1+\eps{X})^{2J}}
  \left[
    \ln\frac{\det\left[\widehat{\mathcal M}^{(X)}_J+\delta\mathcal M_J^{(X)}
      \right]}{\det \widehat{\mathcal M}^{(X)}_J}
  \right]_{\mathcal O(\delta\mathcal M_J^2)}.
  \label{eq_regDiv}
\end{align}

We obtain the results based on the dimensional regularization by
replacing
\begin{align}
  \left[
    \ln\mathcal A^{(X)}
  \right]_{{\rm div}, \eps{X}}
  \rightarrow
  \left[
    \ln\mathcal A^{(X)}
  \right]_{{\rm div}, \epsD}.
\end{align}
The expressions of $[\ln\mathcal A^{(X)}]_{{\rm div}, \eps{X}}$ and
$[\ln\mathcal A^{(X)}]_{{\rm div}, \epsD}$ for the fields of our
interest are given in Appendix \ref{apx_divpart}; the results are
converted to those in the dimensional regularization by the following
replacements:
\begin{itemize}
\item for the effect of the scalar field $\sigma$:
  \begin{align}
    \left(
      \frac{1}{\eps{\sigma}^2}+\frac{2}{\eps{\sigma}}
      +\frac{\kappa}{3|\lambda|}\ln\eps{\sigma}
    \right)
    \rightarrow
    -1-\frac{\kappa}{3|\lambda|}
    \left(
      \frac{1}{2\epsD}+1+\gamma_E+\ln\frac{\mu R}{2}
    \right),
    \label{eq_ScalarRegulatorRel} 
  \end{align}
\item for the Higgs effect:
  \begin{align}
    \left(
      \frac{1}{\eps{h}^2}+\frac{2}{\eps{h}}-\ln\eps{h}
    \right)
    \rightarrow
    \frac{1}{2\epsD}+\gamma_E+\ln\frac{\mu R}{2}, \\
  \end{align}
\item for the effect of fermions:
  \begin{align}
    & \left(
      \frac{2}{\eps{\psi}^2}+\frac{5}{\eps{\psi}}
      +\frac{1}{3}\frac{y^2+2|\lambda|}{|\lambda|}\ln\eps{\psi}
    \right)\nonumber                                                 \\
    & \hspace{3ex}\to -\frac{y^2}{3|\lambda|}
    \left(
      \frac{1}{2\epsD}+1+\gamma_E+\ln\frac{\mu R}{2}
    \right)-\frac{2}{3}
    \left(
      \frac{1}{2\epsD}+\frac{25}{4}+\gamma_E+\ln\frac{\mu R}{2}
    \right),  
  \end{align}
\item for the effect of gauge and NG fields:
  \begin{align}
    & \left(
      \frac{6g^2}{|\lambda|}-2
    \right)
    \left(
      \frac{1}{\eps{A}^2}+\frac{2}{\eps{A}}
    \right)+
    \left(
      \frac{2}{3}+\frac{2g^4}{|\lambda|^2}
    \right)\ln\eps{A}\nonumber                                       \\
    & \hspace{3ex}\to-2
    \left(
      \frac{1}{3}+\frac{2g^2}{|\lambda|}+\frac{g^4}{|\lambda|^2}
    \right)
    \left(
      \frac{1}{2\epsD}+1+\gamma_E+\ln\frac{\mu R}{2}
    \right)+2+2\frac{g^4}{|\lambda|^2}(10-\pi^2).
  \end{align}
\end{itemize}
Subtracting $1/\epsD$, we get a renormalized prefactor in the
$\msbar$-scheme.


\subsection{Dilatational zero mode}

In the calculation of the decay rate of the EW vacuum, we have an
integral over $\bar\phi_C$ in association with the classical scale
invariance, as we saw in eq.~\eqref{eq_dilatationalZeromode}.  So far,
we have performed a one-loop calculation of the decay rate, based on
which the decay rate behaves as
\begin{align}
  \gamma^{\rm (one\mathhyphen loop)}
  \propto\int d\ln R
%  \frac{( \mu R )^{-\frac{8\pi^2\beta_\lambda^{(1)}}{3|\lambda|^2}}}{R^4},
  \frac{1}{R^4}
  \exp \left[
    - \frac{8\pi^2}{3|\lambda (\mu)|}
    - \frac{8\pi^2 \beta_\lambda^{(1)}}{3|\lambda (\mu)|^2} \ln ( \mu R ) 
  \right],
  \label{gamma(1-loop)}
\end{align}
where $\beta_\lambda^{(1)}$ is the one-loop $\beta$ function of
$\lambda$.  The integral does not converge.  

We expect, however, that the integration may converge once higher
order effects are properly included.  To see the detail of the path
integral over the dilatational zero mode, let us denote the decay rate
as
\begin{align}
  \gamma^{\rm (full)} = \int d\ln R
  \frac{e^{-\mathcal B_{\rm eff}}}{R^4},
\end{align}
where $\mathcal{B}_{\rm eff}$ fully takes account of all the effects of higher
order loops.

In order to discuss how $\mathcal{B}_{\rm eff}$ should behave, it is instructive
to rescale the coordinate variable as
\begin{align}
  \tilde x_\mu \equiv \sqrt{\frac{|\lambda|}{8}}\bar\phi_C x_\mu
  = \frac{x_\mu}{R},
\end{align}
as well as the fields as
\begin{align}
  \tilde \Phi \equiv &\, \bar{\phi}_C^{-1} \Phi,
  \\
  \tilde \sigma \equiv &\, \bar{\phi}_C^{-1} \sigma,
  \\
  \tilde A_\mu \equiv &\, \bar{\phi}_C^{-1} A_\mu,
  \\
  \tilde{c} \equiv &\, \bar{\phi}_C^{-1} c,
  \\
  \tilde{\bar{c}} \equiv &\, \bar{\phi}_C^{-1} \bar{c},
  \\
  \tilde \psi \equiv &\,
  \left(\frac{|\lambda|}{8}\right)^{-1/4} \bar\phi_C^{-3/2}
  \psi.
\end{align}
Using the rescaled fields, all the explicit scales disappear from the
action as a result of scale invariance:
\begin{align}
 \frac{1}{\hbar}
 \int d^4 x {\cal L} = 
 \frac{1}{\tilde{\hbar}}
 \int d^4 \tilde{x} 
 \tilde{\cal L}  \left(
   \frac{\kappa}{|\lambda|}, 
   \frac{y}{\sqrt{|\lambda|}},
   \frac{g}{\sqrt{|\lambda|}}
 \right),
\end{align}
where ${\cal L}$ is the total Lagrangian,
\begin{align}
 \tilde{\hbar} = \frac{|\lambda|}{8}\hbar,
\end{align}
and $\tilde{\cal L}$ includes canonically normalized rescaled fields
and depends only on the combinations $\frac{\kappa}{|\lambda|}$,
$\frac{y}{\sqrt{|\lambda|}}$ and $\frac{g}{\sqrt{|\lambda|}}$.  In
addition, the rescaled bounce solution is given by
\begin{align}
 \frac{1}{1+\tilde x^2}.
 \label{eq_rescaledBounce}
\end{align}

Based on $\tilde{\cal L}$ and $\tilde{\hbar}$, we expect:
\begin{itemize}
\item Only positive powers of $\frac{\kappa}{|\lambda|}$,
  $\frac{y}{\sqrt{|\lambda|}}$ and $\frac{g}{\sqrt{|\lambda|}}$ appear
  in the decay rate since there is no singularity when any of these
  goes to zero.
\item When we renormalize divergences using dimensional
  regularization, we introduce a renormalization scale $\tilde\mu$.
  It is always in a logarithmic function and is related to the
  original renormalization scale as
  \begin{align}
    \tilde\mu = \mu R.
  \end{align}
\item In subtracting zero modes associated with transformations of
  eq.~\eqref{eq_rescaledBounce}, the result should be again a
  polynomial of $\frac{\kappa}{|\lambda|}$,
  $\frac{y}{\sqrt{|\lambda|}}$ and
  $\frac{g}{\sqrt{|\lambda|}}$. Notice that, for each zero mode, we
  have $\sqrt{1/\tilde{\hbar}}$ since eq.~\eqref{eq_dzIntegral}
  implies
  \begin{align}
    \frac{dc_0}{\sqrt{2\pi\tilde{\hbar}}\mathcal N_0}
    = dz
    \sqrt{\frac{\langle\tilde \fnc_0|\tilde \fnc_0\rangle}{2\pi\tilde{\hbar}}}.
  \end{align}
\item Quantum corrections have $\tilde{\hbar}^{\ell -1}$ at the
  $\ell$-th loop since the loop expansion is equivalent to the
  $\tilde{\hbar}$ expansion.
\end{itemize}

Based on the above arguments, 
\begin{align}
 \mathcal B_{\rm eff}
  = 
  \frac{8\pi^2}{3|\lambda (\mu)|}
  + \frac{n_{\rm zero}}{2} \ln\frac{|\lambda (\mu)|}{8}
  + \sum_{\ell =1}^{\infty}
  \left(
    \frac{|\lambda (\mu)|}{8}
  \right)^{\ell-1}\mathcal P_\ell
  \left(
    \frac{\kappa (\mu)}{|\lambda (\mu)|},
    \frac{y (\mu)}{\sqrt{|\lambda (\mu)|}},
    \frac{g (\mu)}{\sqrt{|\lambda (\mu)|}},
    \ln \mu R
  \right),
  \label{B_eff}
\end{align}
where $\mathcal P_\ell$ is the contribution at the $\ell$-loop level,
and $n_{\rm zero}$ is the number of zero modes.

If the effects of higher order loops are fully taken into account,
$\mathcal{B}_{\rm eff}$ should be independent of $\mu$ because the decay
rate is a physical quantity; in such a case, we may choose any value of
the renormalization scale $\mu$.  In the perturbative calculation, the
$\mu$-dependence cancells order-by-order \cite{Endo:2015ixx}; we can
explicitly see the cancellation of the $\mu$-dependence at the
leading-log level (see eq.\ \eqref{gamma(1-loop)}).  In our calculation
so far, however, we only have the one-loop result in which $\mu$
dependence remains.  As indicated in Eq.\ \eqref{B_eff}, the $\mu$
dependence shows up in the form of $\ln^p \mu R$ with $p=1$, $2$,
$\cdots$.  If $|\ln\mu R|\gg 1$, the logarithmic term from higher order
loops may become comparable to the tree-level bounce action and the
perturbative calculation breaks down.  In order to make our one-loop
result reliable, we should take $\mu\sim O(1/R)$.  With such a choice of
$\mu$ (as well as with the use of coupling constants evaluated at the
renormalization scale $\mu$), the integration over the size of the
bounce is dominated by the size with which $|\lambda (1/R)|$ becomes the
largest.  In particular, in the case of the standard model, the
integration over the size of the bounce converges with this prescription
as we show in the following section.

\subsection{Final result}

Here, we summarize the results obtained in the previous subsections
and appendices.  A decay rate with a resummation of the
next-to-leading logarithmic terms is given by
\begin{align}
  \gamma = \int d\ln R \frac{1}{R^{4}}
  \left[
    \mathcal A'^{(h)}\mathcal A^{(\sigma)}\mathcal A^{(\psi)}
    \mathcal A^{(A_\mu,\varphi)}e^{-\mathcal B}
  \right]_{\msbar,~\mu\sim 1/R},
\end{align}
where
\begin{align}
  \left[
    \ln\mathcal A'^{(h)}
  \right]_{\msbar}
  & = -\frac{1}{2}
  \left[
    \frac{3}{2}+12\ln A_G-5\ln\frac{\pi}{3}
    +5\ln\frac{|\lambda|}{8}-6\ln\frac{\mu R}{2}
  \right],
  \label{A'^(h)}
  \\
  \left[
    \ln\mathcal A^{(\sigma)}
  \right]_{\msbar}
  & = -\frac{1}{2}
  \left[
    \mathcal S_B(z_\kappa)-\frac{2\kappa}{|\lambda|}-\frac{2\kappa^2}{3|\lambda|^2}
    \left(
      1+\gamma_E+\ln\frac{\mu R}{2}
    \right)
  \right],                       \\
  \left[
    \ln\mathcal A^{(\psi)}
  \right]_{\msbar}
  & = -\frac{y^4}{3|\lambda|^2}
  \left(
    1+\gamma_E+\ln\frac{\mu R}{2}
  \right)-\frac{2y^2}{3|\lambda|}
  \left(
    \frac{25}{4}+\gamma_E+\ln\frac{\mu R}{2}
  \right)+\mathcal S_F(z_y),   \\
  \left[
    \ln\mathcal A^{(A_\mu,\varphi)}
  \right]_{\msbar}
  & = \ln\mathcal V_G+\sum_a
  \left[
    \ln\mathcal A'^{(A_\mu,\varphi)}
  \right]_{\msbar},
\end{align}
with
\begin{align}
  \left[
    \ln\mathcal A'^{(A_\mu,\varphi)}
  \right]_{\msbar}
  & = -\frac{1}{2}
  \Bigg[
  -2
  \left(
    \frac{1}{3}+\frac{2g^2}{|\lambda|}+\frac{g^4}{|\lambda|^2}
  \right)
  \left(
    1+\gamma_E+\ln\frac{\mu R}{2}
  \right)
  \nonumber
  \\ & 
  +2\frac{g^4}{|\lambda|^2}(10-\pi^2)+\frac{1}{2}
  +\frac{2}{3}\gamma_E+4\ln A_G+\ln\frac{|\lambda|}{8}-\ln\pi J_G^2
  \nonumber
  \\ & 
  +3\mathcal S_B(z_g)+3\ln\Gamma(1-z_g)\Gamma(2+z_g)
  \Bigg] .
  \label{A'(A,NG)}
\end{align}
Here, $A_G$ is the Glaisher number and $\mathcal V_G$ is the volume of
the group space generated by the broken generators.  The definitions
of $\mathcal S_B(z)$, $\mathcal S_F(z)$ can be found in Appendix
\ref{apx_infiniteSum}.  We emphasize that the final result does not
depend on the gauge parameter, $\xi$, and hence our result is gauge
invariant.

%%%%%%%%%%%%%%%%%%%%
%Standard Model
\section{Standard Model}
\label{sec_SM}
\setcounter{equation}{0}

\subsection{Decay rate}

Now, we are at the position to discuss the decay rate of the EW vacuum
in the SM.  As we have discussed, the decay of the EW vacuum is
induced by the bounce configuration whose mass scale is much higher
than the EW scale.  Thus, we approximate the Higgs potential as
\begin{align}
  V(H) = \lambda(H^\dagger H)^2,
  \label{V(Phi)}
\end{align}
where $H= (H^+,H^0)$ is the Higgs doublet in the SM and $\lambda$ is
the Higgs quartic coupling constant.  $\lambda$ depends on the
renormalization scale $\mu$; in the SM, $\lambda$ becomes negative
when $\mu$ is above $10^{10}$ GeV with the best-fit values of the SM
parameters.  In addition, the relevant part of the Yukawa couplings
are given by
\begin{align}
  \mathcal L_{\rm int} \ni
  y_t H \bar{q}_L t_R^{c} + {\rm h.c.},
\end{align}
where $q_L$ and $t_R^c$ are left-handed 3rd generation quark doublet
and right-handed anti-top quark, respectively, and $y_t$ is the top
Yukawa coupling constant.

Assuming that $\lambda<0$, the bounce solution for the SM is given by
\begin{align}
  H|_{\rm bounce} = \frac{1}{\sqrt{2}} e^{i\theta^a \sigma^a}
  \begin{pmatrix}
    0 \\
    \bar\phi(r)
  \end{pmatrix},
\end{align}
where the function $\bar\phi$ obeys Eq.\ \eqref{eq_Bounce}, and is
given by Eq.\ \eqref{BounceSolution}.  In particular, remember that the
bounce solution $\bar\phi$ contains a free parameter, which we choose
$R$, because of the classical scale invariance.

The results given in the previous section can be easily applied to the
case of the SM.  Taking account of the effects of the (physical) Higgs
boson, top quark, and weak bosons (as well as NG bosons),\footnote
%%
{We checked that the effect of the bottom quark is numerically
  unimportant.}
%%
the decay rate of the EW vacuum in the SM can be written in the
following form:\footnote
%%
{We checked that the analytic formulas given below give same results
  as those in \cite{Chigusa:2017dux}, where numerical methods are used
  to sum over the contributions from each $J$, if we take the same
  renormalization scale $\mu$.}
%%
\begin{align}
  \gamma = \int d \ln R
  \frac{1}{R^4}
  \left[
    \mathcal A'^{(h)} \mathcal A^{(t)}
    \mathcal A^{(W,Z,\varphi)}e^{-\mathcal B}
  \right]_{\msbar}.
\end{align}
As we have mentioned, the relevant renormalization scale of the
integrand is $\mu\sim O(1/R)$; in the following numerical analysis, we
take $\mu=1/R$ unless otherwise stated.  If $\lambda(\mu)$ is
positive, there is no bounce solution; the integrand is taken to be
zero in such a case.  In addition, in order to be consistent with our
procedure neglecting the mass term in the Higgs potential, $1/R$
should be much larger than the EW scale.  This condition is
automatically satisfied in the present analysis because $\lambda<0$
occurs at the scale much higher than the electorweak scale.

The Higgs contribution $A'^{(h)}$ is given in Eq.\ \eqref{A'^(h)},
while the top-quark contributions is given by
\begin{align}
  \left[
    \ln \mathcal A^{(t)}
  \right]_{\msbar}
  & = \left.3
    \left[
      \ln \mathcal A^{(\psi)}
    \right]_{\msbar}
  \right|_{y\to y_t},
\end{align}
where the factor of $3$ in the right-hand side is the color factor.

One difference between the case of the $U(1)$ gauge symmetry and the
SM case is the volume of the gauge space.  In the SM case, we may
expand $H$ around the bounce solution with $\theta^a=0$ as
\begin{align}
  H = \frac{1}{\sqrt{2}}
  \begin{pmatrix}
    i (\varphi^1 -i \varphi^2)
    \\
    \bar\phi - i \varphi^3
  \end{pmatrix}.
\end{align}
Here, $\varphi^1$ and $\varphi^2$ are NG bosons absorbed by charged
$W$-bosons while $\varphi^3$ is that by $Z$-boson.  With the change of
$\theta^a$, the NG modes are transformed as
\begin{align}
  \varphi^a \to 
  \varphi^a + \theta^a \tilde{\fnc}_G Y_{0,0,0} + \mathcal O(\theta^2),
\end{align}
where
\begin{align}
  \tilde{\fnc}_G = \sqrt{2\pi^2}J_G\bar\phi,
\end{align}
with the Jacobian being given by $J_G=1$.\footnote
%%
{One may parameterize the bounce configuration using properly
  normalized $SU(2)$ generator $T^a\equiv\sigma^a/2$ as 
  \begin{align*}
    H|_{\rm bounce} = \frac{1}{\sqrt{2}} e^{i\varphi^a T^a}
    \begin{pmatrix}
      0 \\
      \bar\phi(r)
    \end{pmatrix},
  \end{align*}
  with which $J_G=\frac{1}{2}$ and the volume of the gauge space is
  \begin{align}
    \int d^3 \vartheta = 16\pi^2.
  \end{align}
  The final result of the decay rate does not change for such a
  convention.}
%%
The integration over the zero modes in association with
the gauge transformation of the bounce solution can be replaced by the 
integration over the parameter $\theta^a$:
\begin{align}
  \int d^3 \theta = 2\pi^2 \equiv {\cal V}_{SU(2)}.
\end{align}
Then, following the argument given in Appendix \ref{apx_zeromode}, the
gauge contribution is evaluated as
\begin{align}
  \left[
    \ln \mathcal A^{(W,Z,\varphi)}
  \right]_{\msbar}
  = &\, 
  \ln\mathcal V_{SU(2)}
  + 2
  \left. 
    \left[
      \ln \mathcal A'^{(A_\mu,\varphi)}
    \right]_{\msbar}
  \right|_{g\to g_W}
  + 
  \left. 
    \left[
      \ln \mathcal A'^{(A_\mu,\varphi)}
    \right]_{\msbar}
  \right|_{g\to g_Z},
\end{align}
where $\mathcal A'^{(A_\mu,\varphi)}$ is given in Eq.\
\eqref{A'(A,NG)}, and
\begin{align}
 g_W = \frac{g_2}{2},~
 g_Z = \frac{\sqrt{g_Y^2+g_2^2}}{2}.
\end{align}
with $g_2$ and $g_Y$ being the gauge coupling constants of $SU(2)_L$
and $U(1)_Y$, respectively.

\subsection{Numerical results}

Now, let us numerically evaluate the decay rate of the EW vacuum in
the SM.\footnote
%%
{The analysis is basically the same as in \cite{Chigusa:2017dux}, but
  we use analytic formulae and a different renormalization scale.}
%%
The decay rate of the EW vacuum is very sensitive to the coupling
constants in the standard model.  In our numerical analysis, we
evaluate them using \cite{Patrignani:2016xqp}:
\begin{align}
 m_h & = 125.09\pm0.24,   \\
 m_t^{\rm (pole)} & = 173.1\pm0.6,     \\
 \alpha_s (m_Z)  & = 0.1181\pm0.0011,
\end{align}
where $m_h$ and $m_t^{\rm (pole)}$ are Higgs mass and top mass,
respectively, while $\alpha_s$ is the strong coupling constant.  The
$SU(2)_L$ and $U(1)_Y$ gauge coupling constants are evaluated by using
$\sin\theta_W=???$ and $\alpha_{\rm em} (m_Z)=???$, where $\theta_W$ is
the Weinberg angle and $\alpha_{\rm em}$ is the fine structure constant.
\rem{SC: According to \cite{Buttazzo:2013uya}, $g_2$ and $g_Y$ are
determined at $\mu = m_t^{\rm pole}$ using $m_W^{\rm pole}$, $m_Z^{\rm
pole}$, $m_h^{\rm pole}$, $m_t^{\rm pole}$, $G_\mu$, and $\alpha_s
(m_Z^{\rm pole})$ as inputs.  Calculations are done in the on-shell
scheme at NNLO precision.}

First, we show the RG evolution of the SM coupling constants in fig.\
\ref{fig_rge}. We use three-loop beta functions given in
\cite{Buttazzo:2013uya} \rem{Maybe more refs????} and the central
values for the SM parameters. The black dotted line indicates where
$\bar\phi_C$ reaches the Planck scale $M_{\rm Pl}\simeq 2.4\times
10^{18}\ {\rm GeV}$.  We also show the running above the Planck scale,
assuming there are no significant corrections from gravity.\footnote
%%
{\rem{Comment on the effects of Planck suppressed operator, and
    references (maybe a lot of).}
  \cite{Burgess:2001tj,Branchina:2015nda,Branchina:2016bws,
    Branchina:2014rva, Branchina:2014usa,Branchina:2013jra}}
%%
For $10^{10}\ {\rm GeV}\lesssim \mu \lesssim10^{30}\ {\rm GeV}$,
$\lambda$ becomes negative; for such a region, we use a dashed line to
indicate $\lambda<0$.

In order to understand the $\mu$ dependence of $\lambda$, let us show
one-loop RG equations of $\lambda$ and $y_t$ (although, in our
numerical calculation, we use RG equations including two- and
three-loop effects):
\begin{align}
 16\pi^2\frac{d\lambda}{d\ln\mu}
  & = 12\lambda
 \left(
  2\lambda+y_t^2-\frac{g_Y^2+g_2^2}{4}-\frac{g_2^2}{2}
 \right)-6y_t^4+6
 \left(
  \frac{g_Y^2+g_2^2}{4}
 \right)^2+12
 \left(
  \frac{g_2^2}{4}
 \right)^2,     \\
 16\pi^2\frac{dy_t}{d\ln\mu}
  & = y_t
 \left(
  \frac{9}{2}y_t^2-8g_3^2-\frac{9}{4}g_2^2-\frac{17}{12}g_Y^2
 \right).
\end{align}
At a low energy scale, the term proportional to $y_t^4$ drives
$\lambda$ to a negative value. As the scale increases, $y_t$ decreases
while $g_Y$ increases, which brings $\lambda$ back to a positive
value. Notice that $\lambda$ is bounded from below in the SM.

\begin{figure}[t]
 \begin{center}
  \hspace{0.8ex}\includegraphics[width=0.6\linewidth]{rge.eps}
  \includegraphics[width=0.625\linewidth]{lngam.eps}
\end{center}
\caption{{\it Top}: The RG evolution of the relevant SM coupling
  constants as a function of $1/R$ (with $R$ being in units of ${\rm
    GeV}^{-1}$), taking $\mu=1/R$.  The dashed line indicates
  $\lambda<0$. The black dotted line shows the scale where
  $\bar\phi_C=M_{\rm Pl}$. The horizontal axis is common with the
  bottom panel. {\it Bottom}: The integrand of the decay rate. We use
  the central values for the SM parameters. In the shaded region,
  $\lambda$ is positive and the integrand is zero.  \rem{From the
    $x$-label, remove $\ln\mu$!!!}  \rem{SC: I prefer the $y$-label $\ln
 \frac{d \gamma}{d \ln R}$.}}
 \label{fig_rge}
\end{figure}

We show the integrand of $\gamma$ in the bottom panel of fig.\
\ref{fig_rge}, together with that of
\begin{align}
 \gamma_{\rm tree} = \int d \ln R \frac{1}{R^4} e^{-\mathcal B}.
 \label{eq_treelevel}
\end{align}
They are also shown in a linear scale in the top panel of fig.\
\ref{fig_dgamma}.

There are some remarks on the integral over $R$.
\begin{itemize}
\item As indicated by the top panel of fig.\ \ref{fig_dgamma}, the
  integral is dominated by the scale $10^{17}~{\rm
    GeV}\lesssim1/R<10^{18}~{\rm GeV}$, corresponding to $10^{18}~{\rm
    GeV}\lesssim\bar\phi_C<10^{19}~{\rm GeV}$, which is close to the
  Planck scale.  We may formally perform the $R$-integration up to the
  scale where $\lambda$ becomes positive again; result of such an
  analysis is denoted as $\gamma_\infty$.  Otherwise, we may stop the
  integration at $\bar{\phi}_C\sim M_{\rm Pl}$, expecting that the SM
  breaks down at the Planck scale due to an effect of quantum gravity;
  we also perform such an calculation terminating the integral at
  $\bar{\phi}_C=M_{\rm Pl}$, assuming that the bounce solution is
  unaffected by the effect of quantum gravity, and the result is
  denoted as $\gamma_{\rm Pl}$.

\item As one can see in the bottom panel of fig.\ref{fig_rge}, there
  is an artificial divergence of the integrand at $1/R\simeq10^{10}\
  {\rm GeV}$.  This is due to a breakdown of perturbative expansion
  owing to too small $|\lambda|$, which makes the one-loop effect
  larger than the tree-level one.  We expect that the effect of such a
  bounce configuration is unimportant because the bounce action for
  such a small $|\lambda|$ is significantly suppressed.  Thus, we
  exclude such a region from the integration over $R$.  In our
  numerical calculation, we integrate over $R$ only in the region such
  that $\left|\frac{\delta\mathcal B_{\rm eff}^{(1)}}{\mathcal B}\right|
      <0.8$ and $\left|\frac{[\ln \mathcal A^{(X)}]_{\rm
      \overline{MS}}}{\mathcal B}\right| <0.8$ for each $X$, where
      $\delta \mathcal B_{\rm eff}^{(1)}$ is the one-loop contribution
      to $\mathcal{B}_{\rm eff}$, while $[\ln \mathcal A^{(X)}]_{\rm
      \overline{MS}}$ is a contribution from particle $X$.  \rem{Second
      condition???? zero-mode????}

\end{itemize}

% Notice that, since we directly integrate the direction of dilatation,
% instead of the saddle point approximation, it does not matter if there
% is no maximum of the integrand in the region of integration. When the
% maximum is outside of the region, what we obtain is a lower limit of the
% decay rate since the integrand is positive definite.

By numerically integrating over $R$, we obtain\footnote
%%
{If we use $\mu=\bar\phi_C$ instead of $1/R$, the decay rate is given
  by $\log_{10}\left[\gamma\times{\rm Gyr~Gpc^3}\right]=-554$,
  reproducing the result of \cite{Chigusa:2017dux}.}
%%
\begin{align}
  \log_{10}\left[\gamma_{\rm Pl}\times{\rm Gyr~Gpc^3}\right]
  & = -564^{+38~+177~+139~+2}_{-43~-312~-208~-2}, \\
  \log_{10}\left[\gamma_{\infty}\times{\rm Gyr~Gpc^3}\right]
  & = -563^{+38~+177~+139~+1}_{-43~-313~-209~-3},
\end{align}
where the 1st, 2nd, 3rd, and 4th errors are due to the Higgs mass, the
top mass, the strong coupling constant, and the renormalization scale,
respectively.  (In order to estimate the uncertainty due to the choice
of the renormalization scale, we vary the renormalization scale from
$1/(2R)$ to $2/R$.)  Since $H_0^{-4}\simeq10^3{\rm Gyr~Gpc^3}$, the
lifetime of the EW vacuum within our Hubble volume is much longer than
the age of the universe. 

\begin{figure}[t]
 \begin{center}
  \hspace{1.6ex}\includegraphics[width=0.6\linewidth]{gam.eps}
  \includegraphics[width=0.6485\linewidth]{lnA.eps}
 \end{center}
 \caption{{\it Top}: The integrand of the decay rate with the central SM
   parameters. The solid line corresponds to a result at the one-loop
   level and the dashed one corresponds to that at the tree level. The
   horizontal axis is common with the bottom figure. We show $\bar\phi_C
   = M_{\rm Pl}$ with the vertical dotted line.  \rem{SC: I prefer the
   $y$-label $\frac{d \gamma}{d \ln R}$.}  {\it Bottom}: The size of
   each quantum correction. The dashed line corresponds to
   $\delta\mathcal B^{(1)}_{\rm eff}$.}  \label{fig_dgamma}
\end{figure}

In fig.\ \ref{fig_SMconst}, we show the decay rate in $m_h$ vs.\
$m_t$ plane.  In the red region, $\gamma$ becomes larger than $H_0^4$,
which we call unstable region.  In the yellow region, the EW vacuum is
metastable, meaning that $0<\gamma<H_0^4$.  In the green region, the
EW vacuum is absolutely stable because $\lambda$ is always positive.
The dashed, solid, and dotted lines correspond to $\alpha_s =
0.1192,~0.1181$, and $0.1170$, respectively. The red
contours show $\ln\left[\gamma\times{\rm Gyr~Gpc^3}\right]=-100$,
$-300$, and $-1000$ with the central value of $\alpha_s$.  \rem{Make
  the contour black!!!}  We also show $1\sigma,$ $2\sigma$ and
$3\sigma$ constraints on the Higgs mass and the top quark mass with
the blue regions, adding their errors in quadrature.

\rem{TM: I could not understand what the following paragraph
  means?????}

We cut-off the integral at $\bar\phi_C = M_{\rm Pl}$, but it does not
change the figure since the maximum is not so far from the Planck
scale.\footnote
%%
{Even with a lower cut-off such as $\bar\phi_C=0.1M_{\rm Pl}$, the
  result does not change significantly.}
%%
The value of $\bar\phi_C$ at the maximum of the integrand ranges from
$10^{18}$ GeV to $10^{20}$ GeV in the figure.

For comparison, we also perform a ``tree-level'' calculation of the
decay rate using eq.~\eqref{eq_treelevel}.  The results are
$\log_{10}\left[\gamma_{\rm Pl}^{\rm (tree)}\times{\rm Gyr~Gpc^3}\right]
= -573$ and $\log_{10}\left[\gamma_{\infty}^{\rm (tree)}\times{\rm
Gyr~Gpc^3}\right] = -570$.  Thus, the difference between $\gamma$ and
$\gamma^{\rm (tree)}$ turns out to be rather small.  This is a
consequence of an accidental cancellation among the contributions of
several fields.  In the bottom panel of fig.\ \ref{fig_dgamma}, we show
individual quantum corrections separately, as well as the total one-loop
contribution.  We can see that a significant cancellaton among the
quantum corrections occurs around $10^{17}~{\rm
GeV}\lesssim1/R<10^{18}~{\rm GeV}$.  We have also checked that the
unstable region on the $m_h$ vs.\ $m_t$ plane shifts upward by $\Delta
m_t\simeq0.25\ {\rm GeV}$ if we use $\gamma^{\rm (tree)}$.

\begin{figure}[t]
 \begin{center}
  \includegraphics[width=0.5\linewidth]{sm.eps}
 \end{center}
 \caption{The stability of the EW vacuum in the SM with a cut-off of the
 integration at $\bar\phi_C=M_{\rm Pl}$. The red region is unstable, the
 yellow region is meta-stable, and the green region is absolutely
 stable. The dashed, solid, and dotted lines correspond to $\alpha_s =
 0.1192,~0.1181$, and $0.1170$, respectively. The red
 contours indicate $\ln\left[\gamma\times{\rm
   Gyr~Gpc^3}\right]=-100,-300$, and $-1000$ with the central value of
 $\alpha_s$. The blue circles indicate $1\sigma,~2\sigma$ and $3\sigma$
 constraints on the Higgs and the top quark masses, with an assumption
 of Gaussian distribution for errors.}  \label{fig_SMconst}
\end{figure}

%%%%%%%%%%%%%%%%%%%%
%extra matter
\section{Models with Extra Fermions}
\label{sec_extra}
\setcounter{equation}{0}

In this section, we consider extensions of the SM with introducing
extra fermions.

\rem{Motivation: PQ, seesaw, other new physics???}


We perform the RG analysis of the runnings of coupling constants with
including the effects of extra fermions.  We include two-loop effects of
extra fermions into the $\beta$-functions, which can be calculated using
the result in \cite{Luo:2002ti}.  We also take account of one-loop
threshold corrections due to the extra fermions, which are summarized in
Appendix \ref{apx_threshold}.\footnote
%%
{If we use the two-loop beta functions instead of three-loop ones in the
  SM calculation, the difference of $\log_{10}\left[\gamma\times{\rm
  Gyr~Gpc^3}\right]$ is around 40. Thus, the systematic error of
  neglecting three-loop effects of the extra fermions is expected to be
  similar or less.}
%%
For the integration over $R$, we follow the procedure in the SM case,
as well as the following treatments:
\begin{itemize}
\item We terminate the integration if any of the coupling constants
  (in particular, Yukawa coupling constants of extra particles)
  exceeds $\sqrt{4\pi}$ since the RG evolution is not reliable above
  such a scale.  \rem{Which region of the figure???  Comment that the
    model is excluded before Yukawa becomes so large????}
\item In order to maintain the classical scale invariance at a good
  accuracy, we require $1/R>10M_{\rm ex}$, where $M_{\rm ex}$ is the
  mass scale of the new particles.
\end{itemize}

\subsection{Vector-like fermions}

Here, we consider two examples of vector-like fermions, one is colored
vector-like fermions and the other is non-colored ones. For simplicity,
we assume that the extra fermions have a Yukawa coupling with the SM
Higgs boson.  \rem{SC: What is ``For simplicity'' ??}  (We also assume
that the mixing between the extra fermions and the SM fermions is
negligible.)  Since the additional Yukawa couplings tends to drive
$\lambda$ to a negative value, the decay rate of the EW vacuum typically
becomes larger than that in the SM. It also shifts higher the scale that
minimizes $\lambda$.

We first discuss the colored vector-like fermions, having the same SM
gauge quantum numbers as left-handed quark doublet and right-handed
down quark, as well as their vector-like partners; we add
$Q_{\rm ex}$ $({\bf 3}, {\bf 2},\frac{1}{6})$, 
$\overline{Q}_{\rm ex}$ $(\bar{\bf 3}, {\bf 2},-\frac{1}{6})$, 
$D_{\rm ex}$ $({\bf 3}, {\bf 1},-\frac{1}{3})$, and
$\overline{D}_{\rm ex}$ $(\bar{\bf 3}, {\bf 1},\frac{1}{3})$. 
(In the parenthesis, we show the quantum numbers of $SU(3)_C$,
$SU(2)_L$, and $U(1)_Y$.)   The Lagrangian is given by
\begin{align}
  \mathcal L = \mathcal L_{\rm SM}
  +Y_{\rm ex}\Phi^* Q_{\rm ex}\overline{D}_{\rm ex}
  +\overline{Y}_{\rm ex}\Phi\overline{Q}_{\rm ex}D_{\rm ex}
  +M_{\rm ex}\overline{Q}_{\rm ex}Q_{\rm ex}
  +M_{\rm ex}\overline{D}_{\rm ex}D_{\rm ex},
\end{align}
where $M_{\rm ex}$ is the mass parameter of the extra fermions; for
simplicity, we assume that the mass is universal for extra fermions.
In addition, $Y_{\rm ex}$ and $\overline{Y}_{\rm ex}$ are new Yukawa
coupling constants.  We take these new parameters real and positive.
We set renormalization conditions as
\begin{align}
  Y_{\rm ex}|_{\mu=M_{\rm ex}}
  = \overline{Y}_{\rm ex}|_{\mu=M_{\rm ex}}
  \equiv y_{\rm ex}.
  \label{eq_RGEboundary}
\end{align}

\rem{TM: I prefer to use $Y_D$ or $Y_D^{\rm (extra)}$ instead of
  $Y_{ex}$ here, and $Y_E^{\rm (extra)}$ in the next example.  Same
  for the mass parameter.}

The calculation of the decay rate is parallel to the SM case, and the
decay rate is given in the following form:
\begin{align}
 \gamma = \int d\ln R \frac{1}{R^4}
 \left[
  \mathcal A'^{(h)}\mathcal A^{(t)}\mathcal A^{(A_\mu,\varphi)}
  \mathcal A^{(Q_{\rm ex},\overline{D}_{\rm ex})}
  \mathcal A^{(\overline{Q}_{\rm ex},D_{\rm ex})}
  e^{-\mathcal B}
 \right]_{\msbar,~\mu=1/R},
\end{align}
where $\mathcal A^{(Q_{\rm ex},\overline{D}_{\rm ex})}$ and $\mathcal
A^{(\overline{Q}_{\rm ex},D_{\rm ex})}$ are effects of the extra
fermions on the prefactor:
\begin{align}
 \left[
  \ln \mathcal A^{(Q_{\rm ex},\overline{D}_{\rm ex})}
 \right]_{\msbar}
  & = \left.3
 \left[
  \ln \mathcal A^{(\psi)}
 \right]_{\msbar}
 \right|_{y\to Y_{\rm ex}}, \\
 \left[
  \ln \mathcal A^{(\overline{Q}_{\rm ex},D_{\rm ex})}
 \right]_{\msbar}
  & = \left.3
 \left[
  \ln \mathcal A^{(\psi)}
 \right]_{\msbar}
 \right|_{y\to \overline Y_{\rm ex}}.
\end{align}

In fig.\ \ref{fig_VLQ}, we show the contours of the constant decay rate
on $M_{\rm ex}$ vs.\ $y_{\rm ex}$ plane.  Here, we use the central
values of the SM parameters. The meanings of the shading colors are the
same as in the SM case. The left and the right panels show the results
with and without imposing the condition $\bar{\phi}_C<M_{\rm Pl}$ in
integrating over $R$. As we can see, the effect of such a cut-off is
significant. This is because $\bar\phi_C$ at the maximum of the
integrand, $\bar\phi_C^{\max}$, may become much larger than the Planck
scale in the case with extra fermons, as indicated in the left panel of
fig.\ \ref{fig_cutoffVLF}. To illustrate the cut-off dependence of the
decay rate, we show the constraint with terminating the integration at
$\bar\phi_C=0.1M_{\rm Pl}$ in the left panel (dashed line).  In
addition, when $y_{\rm ex}$ and $M_{\rm ex}$ are small, we have a region
of absolute stability. This is because the addition of colored particles
makes the strong coupling constant larger than the SM case. It rapidly
drives $y_t$ to a small value, which makes $\lambda$ always positive.
Requiring that the lifetime of the EW vacuum should be longer than the
age of the universe, we obtain $y_{\rm ex}\lesssim0.4-0.5$ for
$10^3~{\rm GeV}\lesssim M_{\rm ex}\lesssim10^{15}~{\rm GeV}$.

\begin{figure}[t]
 \begin{minipage}{0.49\linewidth}
  \begin{center}
   \includegraphics[width=\linewidth]{VLQ_C.eps}
  \end{center}
 \end{minipage}
 \begin{minipage}{0.49\linewidth}
  \begin{center}
   \includegraphics[width=\linewidth]{VLQ.eps}
  \end{center}
 \end{minipage}
 \caption{The decay rate of the EW vacuum for vector-like quarks with
 (left) and without (right) the cut-off at the Planck scale. The solid
 lines show $\log_{10}[\gamma~{\rm Gyr}~{\rm
 Gpc}^3]=-250,-500$, and $-1000$. The dashed line corresponds to the constraint
 when we stop the integration at $\bar\phi_C=0.1M_{\rm Pl}$.}
 \label{fig_VLQ}
\end{figure}

The second example is non-colored extra fermions, having the same SM quantum
numbers as leptons.
We introduce
$L_{\rm ex}$ $({\bf 1}, {\bf 2},-\frac{1}{2})$, 
$\overline{L}_{\rm ex}$ $({\bf 1}, {\bf 2},\frac{1}{2})$, 
$E_{\rm ex}$ $({\bf 1}, {\bf 1},-1)$, and
$\overline{E}_{\rm ex}$ $({\bf 1}, {\bf 1},1)$, and the Lagrangian is
given by
\begin{align}
 \mathcal L = \mathcal L_{\rm SM}
 +Y_{\rm ex}\Phi^* L_{\rm ex}\overline{E}_{\rm ex}
 +\overline{Y}_{\rm ex}\Phi\overline{L}_{\rm ex}E_{\rm ex}
 +M_{\rm ex}\overline{L}_{\rm ex}L_{\rm ex}
 +M_{\rm ex}\overline{E}_{\rm ex}E_{\rm ex}.
\end{align}
We adopt the following renormalization conditions
\begin{align}
  Y_{\rm ex}|_{\mu=M_{\rm ex}}
  = \overline{Y}_{\rm ex}|_{\mu=M_{\rm ex}}
  \equiv y_{\rm ex}.
  \label{eq_RGEboundary}
\end{align}
The decay rate of the EW vacuum is given by
\begin{align}
  \gamma = \int d(\ln R)\frac{1}{R^4}
  \left[
    \mathcal A'^{(h)}\mathcal A^{(t)}\mathcal A^{(A_\mu,\varphi)}
    \mathcal A^{(L_{\rm ex},\overline{E}_{\rm ex})}
    \mathcal A^{(\overline{L}_{\rm ex},E_{\rm ex})}
    e^{-\mathcal B}
  \right]_{\msbar,~\mu=1/R},
\end{align}
where
\begin{align}
  \left[
    \ln \mathcal A^{(L_{\rm ex},\overline{E}_{\rm ex})}
  \right]_{\msbar}
  & = \left.
    \left[
      \ln \mathcal A^{(\psi)}
    \right]_{\msbar}
  \right|_{y\to Y_{\rm ex}}, \\
  \left[
    \ln \mathcal A^{(\overline{L}_{\rm ex},E_{\rm ex})}
  \right]_{\msbar}
  & = \left.
    \left[
      \ln \mathcal A^{(\psi)}
    \right]_{\msbar}
  \right|_{y\to \overline Y_{\rm ex}}.
\end{align}

In fig.\ \ref{fig_VLL}, we show the contours of constant decay
rate. Since the extra fermions are not colored, we do not have an
absolutely stable parameter region. We observe a larger effect of the
cut-off at the Planck scale. This is because $\bar\phi_C^{\max}$ is
typically large in a wider parameter space, as indicated in the right
panel of fig.\ \ref{fig_cutoffVLF}.

Requiring that the lifetime of the EW vacuum should be longer than the
age of the universe, we obtain $y_{\rm ex}\lesssim0.4-0.7$ for
$10^3~{\rm GeV}\lesssim M_{\rm ex}\lesssim10^{15}~{\rm GeV}$. The
constraint becomes significantly weaker for larger $M_{\rm ex}$ owing to
the cut-off at the Planck scale.

\begin{figure}[t]
 \begin{minipage}{0.49\linewidth}
  \begin{center}
   \includegraphics[width=\linewidth]{VLL_C.eps}
  \end{center}
 \end{minipage}
 \begin{minipage}{0.49\linewidth}
  \begin{center}
   \includegraphics[width=\linewidth]{VLL.eps}
  \end{center}
 \end{minipage}
 \caption{The same figure as fig.\ \ref{fig_VLQ} but for vector-like
  leptons.}  \label{fig_VLL}
\end{figure}

\begin{figure}[t]
 \begin{minipage}{0.49\linewidth}
  \begin{center}
   \includegraphics[width=\linewidth]{VLQ_max.eps}
  \end{center}
 \end{minipage}
 \begin{minipage}{0.49\linewidth}
  \begin{center}
   \includegraphics[width=\linewidth]{VLL_max.eps}
  \end{center}
 \end{minipage}
 \caption{The value of $\bar\phi_C$ at the maximum of the integrand,
 $\bar\phi_C^{\max}$. The left panel is for vector-like quarks and the
 right is for vector-like leptons. The numbers in the legends indicate
 $\log_{10}[\bar\phi_C^{\max}/{\rm GeV}]$. In the gray region, the EW
 vacuum is absolutely stable.  \rem{SC: The gradation in the upper half
 part of each figure is due to the termination of $R$ integration before
 $\bar{\phi}_C \sim 10^{35}\ {\rm GeV}$, I guess??  It might be
 confusing but we can leave some comments here.}}  \label{fig_cutoffVLF}
\end{figure}


\subsection{Right-handed neutrino}

Next, we consider the case with right-handed neutrinos which is
responsible for the active neutrino masses via the seesaw mechanism
\cite{??????}.  For simplicity, we concentrate on the case where only
one mass eigenstate of the right-handed neutrinos, denoted as $N_R$,
strongly couples to the Higgs boson (as well as to the third generation
lepton doublet).  Then, the Lagrangian is given by
\begin{align}
  \mathcal L = \mathcal L_{\rm SM}
  +Y_N\Phi^* L_3 \overline{N}_R
  +\frac{1}{2}M_N\overline{N}_RN_R,
\end{align}
where, in this subsection, $L_3$ denotes the third generation lepton
doublet. We define
\begin{align}
  Y_N|_{\mu=M_N} \equiv y_{\rm ex}.
  \label{eq_RGEboundary}
\end{align}

Assuming that, for simplicity, the neutrino Yukawa matrix is diagonal
in the mass-basis of right-handed neutrinos, the following effective
operator shows up by integrating out $N_R$:
\begin{align}
 \Delta\mathcal L = \frac{C}{4}(\Phi L_3)^2,
\end{align}
with
\begin{align}
  C(M_N) = -2\frac{y_N^2}{M_N}.
\end{align}
One of the active neutrino masses is related to the value of $C$ at
the EW scale as
\begin{align}
  m_\nu = C(m_t)\frac{v^2}{4},
\end{align}
with $v\simeq 246\ {\rm GeV}$ is the vacuum expectation value of the
Higgs boson.  In our numerial calculation, we use the following
one-loop RGE \cite{Chankowski:1993tx}
\begin{align}
  16\pi^2\frac{d}{d\ln\mu}C = \left(4\lambda-6y_t^2-3g_2^2\right)C.
\end{align}
to estimate the neutrino mass.

In the SM with right-handed neutrinos, the decay rate of the EW vacuum
is evaluated as
\begin{align}
 \gamma = \int d(\ln R)\frac{1}{R^4}
 \left[
  \mathcal A'^{(h)}\mathcal A^{(t)}\mathcal A^{(A_\mu,\varphi)}
  \mathcal A^{(L,N_R)}e^{-\mathcal B}
 \right]_{\msbar,~\mu=1/R}.
\end{align}
where
\begin{align}
 \left[
  \ln \mathcal A^{(L,N_R)}
 \right]_{\msbar}
  & = \left.
 \left[
  \ln \mathcal A^{(\psi)}
 \right]_{\msbar}
 \right|_{y\to Y_{N}}.
\end{align}
In fig.\ \ref{fig_RN}, we show the contour plots of the decay
rate. Since it does not have any SM charges, the decay rate goes to
the SM value when $y_N$ goes to zero.  The effect of the cut-off at
the Planck scale is again large, which is because of a large
$\bar\phi_C^{\max}$ as shown in fig.\ \ref{fig_cutoffRHN}.  The purple
solid lines show the left-handed neutrino mass.  Requiring that the
decay rate should be smaller than the age of the universe, we obtain
$y_{\rm ex}\lesssim0.7-0.8$ for $10^{12}~{\rm GeV}\lesssim m_{\rm
  ex}\lesssim10^{15}~{\rm GeV}$.

\begin{figure}[t]
 \begin{minipage}{0.49\linewidth}
  \begin{center}
   \includegraphics[width=\linewidth]{RHN_C.eps}
  \end{center}
 \end{minipage}
 \begin{minipage}{0.49\linewidth}
  \begin{center}
   \includegraphics[width=\linewidth]{RHN.eps}
  \end{center}
 \end{minipage}
 \caption{The same figure as fig.\ \ref{fig_VLQ} but for a right-handed
  neutrino. We also show lines indicating $m_\nu=0.05$ eV and
  $m_\nu=0.08$ eV with purple solid lines.}  \label{fig_RN}
\end{figure}

\begin{figure}[t]
 \begin{center}
  \includegraphics[width=0.5\linewidth]{RHN_max.eps}
 \end{center}
 \caption{The same figure as fig.\ \ref{fig_cutoffVLF} but for a
   right-handed neutrino. \rem{SC: I want the same figure without $*.5$
 on the x-axis.} }  \label{fig_cutoffRHN}
\end{figure}

%%%%%%%%%%%%%%%%%%%%
%Conclusion
\section{Conclusion}
\label{sec_conclusion}
\setcounter{equation}{0}

\rem{TM: I will check this section after everything will be fixed.}

In this paper, we have evaluated vacuum decay rates for the SM and
various models beyond, with including the complete one-loop corrections
and a resummation of next-to-leading logarithmic terms. In particular, we
have considered models with extra fermions. Since they typically make
the EW vacuum unstable, the constraints on their masses and couplings
are phenomenologically important. We have also presented a complete set
of analytic formulae, which can be applied to any model that exhibits
classical conformality at a high energy scale. Together with Appendix
\ref{apx_recipe}, we can evaluate the decay rate almost instantaneously.

We used a recently developed technique to calculate functional
determinants in a gauge sector, which not only gives a prescription to
integrate gauge zero modes but also allows us to calculate the
determinants analytically. We have obtained a manifestly gauge invariant
result, starting with a Lagrangian that includes a gauge parameter.

To sum up all the effects of bounce solutions with different sizes, one
needs to go beyond the one-loop level, otherwise the decay rate would
become infinite. To improve the result, we resummed the next-to-leading
logarithmic terms using the RG evolution of the couplings. There, we
have found that an appropriate renormalization scale for the resummation
is given by a typical spacial size of a bounce solution.

In the calculation of the decay rate in the SM, we have found that the
field value of the bounce reaches the Planck scale at where
$d\gamma/d(\ln R)$ is maximized. We put a cut-off of the integration so
that the field value does not exceed the Planck scale since we expect
non-negligible corrections from gravity. Then, the result should be
understood as a lower bound on the decay rate. We have obtained
\begin{align}
 \log_{10}\left[\gamma_{\rm Pl}\times{\rm Gyr~Gpc^3}\right]
  & = -564^{+38~+177~+139~+2}_{-43~-312~-208~-2},
\end{align}
where the errors come from the Higgs mass, the top mass, the strong
coupling constant, and the renormalization scale, respectively. The
effect of the cut-off is turned out to be smaller than that of the
renormalization scale. Since $H_0^{-4}\simeq10^3~{\rm Gyr~Gpc^3}$, the
lifetime is long enough compared with the age of the universe.

We have analyzed the decay rate for extensions of the SM with extra
fermions; (i) vector-like quarks, (ii) vector-like leptons, and (iii) a
right-handed neutrino.  We have obtained a constraint on the parameter
space for each model, requiring that the lifetime of the EW vacuum is
longer than the age of the universe. The results constrain the Yukawa
couplings that are larger than about $0.4-0.5$ if we do not consider the
cut-off at the Planck scale. The effect of the cut-off was found to be
rather large and the constraints on the Yukawa couplings become weaker,
at most, by $0.3$ after including the cut-off.

\section*{Acknowledgements}
 This work was supported by the Grant-in-Aid for Scientific Research C
 (No.26400239), and Innovative Areas (No.16H06490). The work of S.C. was
 also supported in part by the Program for Learning Graduate Schools,
 MEXT, Japan.

%Appendix

\appendix
%%%%%%%%%%%%%%%%%%%%
%Functional determinants
\section{Functional Determinant}
\label{apx_determinantJ}
\setcounter{equation}{0}

In this appendix, we present explicit forms of functional determinants
for various fields and evaluate them analytically.  For simplicity, we
consider the case with $U(1)$ gauge interaction; the charge of the
scalar field $\Phi$ responsible for the vacuum instability is taken to
be $+1$.  The potential of $\Phi$ is given by
\begin{align}
  V(\Phi) = \lambda(\Phi^\dagger\Phi)^2,
\end{align}
with $\lambda<0$.

\subsection{Scalar contribution}

We consider a real scalar field, $\sigma$, which couples to the Higgs
field as
\begin{align}
  V=\kappa\sigma^2\Phi^\dagger\Phi,
\end{align}
where $\kappa$ is a positive coupling constant.  The contribution to the
prefactor is given by
\begin{align}
  \ln\mathcal A^{(\sigma)}
  = -\frac{1}{2}\ln\frac{\det[-\partial^2+\kappa\bar\phi^2]}{\det[-\partial^2]}.
\end{align}

First, we expand $\sigma$ into partial waves;
\begin{align}
  \sigma(r,\Omega) = \sigma^{J,m_A,m_B}(r)Y_{J,m_A,m_B}(\Omega),
\end{align}
where $Y_{J,m_A,m_B}(\Omega)$ is the hyperspherical function on $S^3$,
and $\Omega$ is a coordinate on $S^3$. Here, $J$ is a non-negative
half-integer and labels the total angular momentum, and $m_A$ and $m_B$
are the azimuthal quantum numbers for the $A$-spin and the $B$-spin of
$so(4)\simeq su(2)_A\times su(2)_B$, respectively. The Laplacian
operator acts on the hyperspherical function as
\begin{align}
 -\partial^2 Y_{J,m_A,m_B}(\Omega) = \frac{L^2}{r^2}Y_{J,m_A,m_B}(\Omega),
\end{align}
where
\begin{align}
 L = \sqrt{4J(J+1)}.
\end{align}

Since the fluctuation operator for partial waves does not depend on
$m_A$ and $m_B$, we have $(2J+1)^2$ degeneracy for each $J$.  Summing up
all the contributions, we have
\begin{align}
 \ln\mathcal A^{(\sigma)}
 = -\frac{1}{2}\sum_{J=0}^{\infty}(2J+1)^2
 \ln\frac{\det[-\Delta_J+\kappa\bar\phi^2]}{\det[-\Delta_J]},
\end{align}
where
\begin{align}
 \Delta_J = \partial_r^2+\frac{3}{r}\partial_r-\frac{L^2}{r^2}.
\end{align}

Using eq.~\eqref{eq_theorem}, we have
\begin{align}
 \frac{\det[-\Delta_J+\kappa\bar\phi^2]}{\det[-\Delta_J]}
 = \lim_{r\to\infty}\frac{f^{(\sigma)}(r)}{r^{2J}},
\end{align}
where
\begin{align}
 (-\Delta_J+\kappa\bar\phi^2)f^{(\sigma)} = 0,
\end{align}
with $\lim_{r\to0}f^{(\sigma)}(r)/r^{2J} = 1$. Notice that
$\Delta_Jr^{2J} = 0$.

The exact solution to the above differential equation is given by
\begin{align}
 f^{(\sigma)} = r^{2J}
 \left(
  1+\frac{|\lambda|}{8}\bar\phi_C^2r^2
 \right)^{1+z_\kappa}{}_2F_1
 \left(
  1+z_\kappa,2(J+1)+z_\kappa,2(J+1),-\frac{|\lambda|}{8}\bar\phi_C^2r^2
 \right),
\end{align}
where ${}_2F_1(a,b,c,z)$ is the hypergeometric function and
\begin{align}
 z_\kappa = -\frac{1}{2}
 \left(
  1-\sqrt{1-8\frac{\kappa}{|\lambda|}}
 \right).
\end{align}
Taking the limit of $r\to\infty$, we have
\begin{align}
 \frac{\det[-\Delta_J+\kappa\bar\phi^2]}{\det[-\Delta_J]}
 = \frac{\Gamma(2J+1)\Gamma(2J+2)}{\Gamma(2J+1-z_\kappa)\Gamma(2J+2+z_\kappa)}.
 \label{eq_scalarDeterminant}
\end{align}

\subsection{Higgs contribution}

Using the bounce solution given in Eq.\ \eqref{BounceSolution}, the
Higgs mode fluctuation is parametrized as
\begin{align}
 \Phi = \frac{1}{\sqrt{2}}
 \left( 
   \bar\phi(r)+h^{J,m_A,m_B}(r)Y_{J,m_A,m_B}(\Omega)
 \right).
\end{align}
Then, the Higgs contribution to the prefactor is given by
\begin{align}
 \ln\mathcal A^{(h)}
  & = -\frac{1}{2}\ln\frac{\det[-\partial^2-3|\lambda|\bar\phi^2]}{\det[-\partial^2]}\nonumber              \\
  & = -\frac{1}{2}\sum_{J=0}^\infty(2J+1)^2\ln\frac{\det[-\Delta_J-3|\lambda|\bar\phi^2]}{\det[-\Delta_J]}.
\end{align}
The procedure is parallel to the scalar case with replacement
$\kappa\to-3|\lambda|$. From eq.~\eqref{eq_scalarDeterminant}, we have
\begin{align}
 \frac{\det[-\Delta_J-3|\lambda|\bar\phi^2]}{\det[-\Delta_J]}
 = \frac{2J(2J-1)}{(2J+3)(2J+2)}.
\end{align}
As we can see, $J = 0$ and $J = 1/2$ have zero modes, which correspond
to the scale invariance and the translational invariance, respectively.

\subsection{Fermion contribution}

Let us consider chiral fermions $\psi_L$ and $\psi_R$ with the
following Yukawa term
\begin{align}
  \mathcal L_{\rm Yukawa} = y \Phi \bar{\psi}_L \psi_R + {\rm h.c.}
\end{align}

The contribution to the prefactor is given by
\begin{align}
  \ln\mathcal A^{(\psi)}
  & = \ln\frac{\det
    \left[
      \cancel\partial+\frac{y}{\sqrt{2}}\bar\phi
    \right]}{\det[\cancel\partial]}
  \nonumber
  \\
  & = \frac{1}{2}\ln\frac{\det
  \left[
   \left(
    \cancel\partial+\frac{y}{\sqrt{2}}\bar\phi
   \right)
   \left(-\cancel\partial+\frac{y}{\sqrt{2}}\bar\phi
   \right)
  \right]}{\det[-\cancel\partial\cancel\partial]}\nonumber \\
  & = \frac{1}{2}\ln\frac{\det
  \left[
   -\partial^2+\frac{y^2}{2}\bar\phi^2+\frac{y}{\sqrt{2}}(\cancel\partial\bar\phi)
  \right]}{\det[-\partial^2]}.
\end{align}
Using the basis given in \cite{Avan:1985eg}, we have
\begin{align}
 \ln\mathcal A^{(\psi)}
  & = \sum_{J=0}^{\infty} (2J+1)(2J+2)
 \ln\frac{\det\mathcal M^{(\psi)}_J}{\det\widehat{\mathcal M}^{(\psi)}_J}.
\end{align}
where
\begin{align}
 \mathcal M^{(\psi)}_J =
 \begin{pmatrix}
  -\Delta_J+\frac{y^2}{2}\bar\phi^2 & \frac{y}{\sqrt{2}}\bar\phi'             \\
  \frac{y}{\sqrt{2}}\bar\phi'       & -\Delta_{J+1/2}+\frac{y^2}{2}\bar\phi^2
 \end{pmatrix},
\end{align}
and $\widehat{\mathcal M}^{(\psi)}_J$ is obtained by replacement
$\bar\phi\to0$ and $\bar\phi'\to0$. Here, $\bar\phi'=d\bar\phi/dr$.

Using eq.~\eqref{eq_theorem}, we have
\begin{align}
 \frac{\det\mathcal M^{(\psi)}_J}{\det\widehat{\mathcal M}^{(\psi)}_J}
 =
 \left(
  \lim_{r\to\infty}\frac{\det[\Psi_1~\Psi_2]}{r^{4J+1}}
 \right)
 \left(
  \lim_{r\to0}\frac{\det[\Psi_1~\Psi_2]}{r^{4J+1}}
 \right)^{-1},
\end{align}
where
\begin{align}
 \mathcal M^{(\psi)}_J\Psi_i(r) = 0.
\end{align}

Solutions can be decomposed into two functions as
\begin{align}
 \Psi_i =
 \begin{pmatrix}
  \frac{\sqrt{2}}{y\bar\phi}\frac{1}{r^{2J+3}}\partial_rr^{2J+3}\chi_i \\
  \chi_i
 \end{pmatrix} +
 \begin{pmatrix}
  -\frac{2}{y^2\bar\phi^2}\frac{1}{r^{2J+3}}\partial_rr^{2J+3}\eta_i \\
  0
 \end{pmatrix},
\end{align}
where
\begin{align}
 \left[
  -\Delta_{J+1/2}
  +\frac{\bar\phi'}{\bar\phi}\frac{1}{r^{2J+3}}\partial_rr^{2J+3}
  +\frac{y^2}{2}\bar\phi^2
 \right]\chi_i
  & =\frac{\sqrt{2}\bar\phi'}{y\bar\phi^2}\frac{1}{r^{2J+3}}\partial_rr^{2J+3}\eta_i, \\
 \left[
  -\Delta_{J+1/2}
  +\frac{\bar\phi'}{\bar\phi}\frac{1}{r^{2J+3}}\partial_rr^{2J+3}
  +\frac{y^2}{2}\bar\phi^2
 \right]\eta_i
  & =0.
\end{align}

For the first solution, we take
\begin{align}
 \chi_1(r) = f^{(\psi)}(r),~\eta_1(r) = 0,
\end{align}
where
\begin{align}
 f^{(\psi)}(r) = r^{2J+1}
 \left(
  1+\frac{|\lambda|}{8}\bar\phi_C^2r^2
 \right)^{-i\frac{y}{\sqrt{|\lambda|}}}{}_2F_1
 \left(
  2J+2-i\frac{y}{\sqrt{|\lambda|}},1-i\frac{y}{\sqrt{|\lambda|}},
  2J+3,-\frac{|\lambda|}{8}\bar\phi_C^2r^2
 \right).
\end{align}
Here, $f^{(\psi)}$ behaves as
\begin{align}
 \lim_{r\to0}\frac{f^{(\psi)}(r)}{r^{2J+1}}
  & = 1,                              \\
 \lim_{r\to\infty}\frac{f^{(\psi)}(r)}{r^{2J-1}}
  & = \frac{8}{|\lambda|\bar\phi_C^2}
 \frac{\Gamma(2J+1)\Gamma(2J+3)}{\Gamma
  \left(
   2J+2-i\frac{y}{\sqrt{\lambda}}
  \right)
  \Gamma
  \left(
   2J+2+i\frac{y}{\sqrt{\lambda}}
  \right)}.
\end{align}

For the second solution, we take
\begin{align}
 \eta_2(r) = f^{(\psi)}(r).
\end{align}
Then, $\chi_2$ behaves as
\begin{align}
 \lim_{r\to0}\frac{\chi_2(r)}{r^{2J+3}}
  & = -\frac{\sqrt{2}\bar\phi''(0)}{y\bar\phi_C^2}\frac{J+1}{2J+3}, \\
 \lim_{r\to\infty}\frac{\chi_2(r)}{r^{2J+1}}
  & = \frac{\sqrt{2}}{y\bar\phi_C}
 \frac{[\Gamma(2J+2)]^2}{\Gamma
  \left(
   2J+2-i\frac{y}{\sqrt{\lambda}}
  \right)\Gamma
  \left(
   2J+2+i\frac{y}{\sqrt{\lambda}}
  \right)}.
\end{align}
Using these solutions, we get
\begin{align}
 \lim_{r\to0}\frac{\det[\Psi_1~\Psi_2]}{r^{4J+1}}
  & = \frac{8(J+1)}{y^2\bar\phi_C^2}, \\
 \lim_{r\to\infty}\frac{\det[\Psi_1~\Psi_2]}{r^{4J+1}}
  & = \frac{8(J+1)}{y^2\bar\phi^2}
 \left[
  \frac{[\Gamma(2J+2)]^2}{\Gamma
   \left(
    2J+2-i\frac{y}{\sqrt{\lambda}}
   \right)\Gamma
   \left(
    2J+2+i\frac{y}{\sqrt{\lambda}}
   \right)}
 \right]^2.
\end{align}
Thus, we have
\begin{align}
 \frac{\det\mathcal M^{(\psi)}_J}{\det\widehat{\mathcal M}^{(\psi)}_J}
 =
 \left[
  \frac{[\Gamma(2J+2)]^2}{\Gamma
   \left(
    2J+2-i\frac{y}{\sqrt{\lambda}}
   \right)\Gamma
   \left(
    2J+2+i\frac{y}{\sqrt{\lambda}}
   \right)}
 \right]^2.
 \label{eq_FermionDeterminant}
\end{align}

\subsection{Gauge contribution}

We consider the contributions from gauge bosons, NG bosons, and the
Faddeev-Popov ghosts.  Let us consider the following Lagrangian;
\begin{align}
  \mathcal L = \frac{1}{4} F_{\mu\nu} F_{\mu\nu}
  + \left[
    \left(
      \partial_\mu - i g A_\mu 
    \right) \Phi
  \right]^\dagger
  \left(
    \partial_\mu - i g A_\mu 
  \right) \Phi
  +V(\Phi)+\mathcal L_{\rm GF}+\mathcal L_{\rm FP},
\end{align}
where $F_{\mu\nu}$ is the field strength tensor, and
\begin{align}
  \mathcal L_{\rm GF} & = \frac{1}{2\xi}\mathcal F^2, \\
  \mathcal L_{\rm FP} & = \bar{c}(-\partial_\mu \partial_\mu)c,
\end{align}
with
\begin{align}
  \mathcal F = \partial_\mu A_\mu.
\end{align}

Since the Faddeev-Popov ghosts do not directly couple to the Higgs
field, we have
\begin{align}
 \ln \mathcal A^{(c,\bar{c})} = 0.
\end{align}

At the one-loop level, $\ln\mathcal A^{(A_\mu,\varphi)}$ is given by
\begin{align}
  \ln\mathcal A^{(A_\mu,\varphi)}
  & = 
  - \frac{1}{2}
  \ln
  \frac{\det\mathcal M^{(A_\mu,\varphi)}}
  {\det\widehat{\mathcal M}^{(A_\mu,\varphi)}},
\end{align}
where
\begin{align}
 \mathcal M^{(A_\mu,\varphi)} =
 \begin{pmatrix}
  -\partial^2\delta_{\mu\nu}+
  \left(
   1-\frac{1}{\xi}
  \right)\partial_\mu\partial_\nu+g^2\bar\phi^2
   & g(\partial_\nu\bar\phi)-g\bar\phi\partial_\nu \\
  2g(\partial_\mu\bar\phi)+g\bar\phi\partial_\mu
   & -\partial^2+m_{a}^2
 \end{pmatrix}.
 \label{eq_gaugeFluct}
\end{align}
Here, $m_{a}^2 = d^2V/d\varphi d\varphi$.

\rem{Some new notation for $m_a$!!!}

For the partial wave expansions, we use the following basis;
\begin{align}
 A_\mu(r,\Omega)
  & = \alpha_S^{J,m_A,m_B}(r)\frac{x_\mu}{r} Y_{J,m_A,m_B}(\Omega)
 +\alpha_L^{J,m_A,m_B}(r)\frac{r}{L}\partial_\mu Y_{J,m_A,m_B}(\Omega)\nonumber \\
  & \hspace{3ex}+\alpha_{T1}^{J,m_A,m_B}(r)i\varepsilon_{\mu\nu\rho\sigma}
 V_\nu^{(1)}L_{\rho\sigma}Y_{J,m_A,m_B}(\Omega)
 +\alpha_{T2}^{J,m_A,m_B}(r)i\varepsilon_{\mu\nu\rho\sigma}
 V_\nu^{(2)}L_{\rho\sigma}Y_{J,m_A,m_B}(\Omega),                                \\
 \varphi(r,\Omega)
  & = \varphi^{J,m_A,m_B}(r)Y_{J,m_A,m_B}(\Omega),
 \label{eq_gaugeBasis}
\end{align}
where $V_\nu^{(i)}$'s are arbitrary independent vectors and
$\varepsilon_{\mu\nu\rho\sigma}$ is a fully anti-symmetric tensor.  We
have
\begin{align}
 \ln\mathcal A^{(A_\mu,\varphi)}
 &= -\frac{1}{2}\sum_{J = 0}^{\infty}(2J+1)^2
 \left[
  \ln\frac{\det\mathcal M^{(A_\mu,\varphi)}_J}{\det\widehat{\mathcal M}^{(A_\mu,\varphi)}_J}
 \right]\nonumber \\
 &= -\frac{1}{2}\ln\frac{\det\mathcal M^{(S,\varphi)}_0}{\det\widehat{\mathcal M}^{(S,\varphi)}_0}
 -\frac{1}{2}\sum_{J = 1/2}^{\infty}(2J+1)^2
 \left[
  \ln\frac{\det\mathcal M^{(S,L,\varphi)}_J}{\det\widehat{\mathcal M}^{(S,L,\varphi)}_J}
  +2\ln\frac{\det\mathcal M^{(T)}_J}{\det\widehat{\mathcal M}^{(T)}_J}
 \right],
 \label{eq_gaugeSLT}
\end{align}
where
\begin{align}
 \mathcal M^{(T)}_J = -\Delta_J+g^2\bar\phi^2,
\end{align}
\begin{align}
 \mathcal M^{(S,L,\varphi)}_J & =
 \begin{pmatrix}
  -\Delta_J+\frac{3}{r^2}+g^2\bar\phi^2
   & -\frac{2L}{r^2}
   & g\bar\phi'-g\bar\phi\partial_r                                  \\
  -\frac{2L}{r^2}
   & -\Delta_J-\frac{1}{r^2}+g^2\bar\phi^2 & -\frac{L}{r}g\bar\phi \\
  2g\bar\phi'+g\bar\phi\frac{1}{r^3}\partial_rr^3
   & -\frac{L}{r}g\bar\phi
   & -\Delta_J+m_a^2
 \end{pmatrix}\nonumber                                     \\
                              & \hspace{3ex}+\left(1-\frac{1}{\xi}\right)
 \begin{pmatrix}
  \Delta_{1/2}
   & -L\partial_r\frac{1}{r}
   & 0                       \\
  \frac{L}{r^4}\partial_rr^3
   & -\frac{L^2}{r^2}
   & 0                       \\
  0
   & 0
   & 0
 \end{pmatrix},
\end{align}
and
\begin{align}
 \mathcal M^{(S,\varphi)}_0 & =
 \begin{pmatrix}
  \frac{1}{\xi}\left(-\Delta_{1/2}+\xi g^2\bar\phi^2\right)
   & g\bar\phi'-g\bar\phi\partial_r \\
  2g\bar\phi'+g\bar\phi\frac{1}{r^3}\partial_rr^3
   & -\Delta_0+m_a^2
 \end{pmatrix}.
\end{align}

Solutions can be constructed from
\cite{Endo:2017gal,Endo:2017tsz}\footnote
%%
{The function $\eta$ in the present analysis corresponds to $\eta/L$
  in \cite{Endo:2017gal,Endo:2017tsz}; such a rescaling makes the
  formulas simpler.}
%%
\begin{align}
 \Psi_i\equiv
 \begin{pmatrix}
  \Psi_i^{\rm (top)} \\
  \Psi_i^{\rm (mid)} \\
  \Psi_i^{\rm (bot)}
 \end{pmatrix}=
 \begin{pmatrix}
  \partial_r\chi_i  \\
  \frac{L}{r}\chi_i \\
  g\bar\phi\chi_i
 \end{pmatrix}
 +
 \begin{pmatrix}
  \frac{L}{r}\frac{1}{g^2\bar\phi^2}\eta_i                \\
  \frac{1}{g^2\bar\phi^2}\frac{1}{r^2}\partial_rr^2\eta_i \\
  0
 \end{pmatrix}
 +
 \begin{pmatrix}
  -\frac{2\bar\phi'}{g^2\bar\phi^3}\zeta_i \\
  0                                          \\
  \frac{1}{g\bar\phi}\zeta_i
 \end{pmatrix},
\end{align}
where
\begin{align}
  & \Delta_J\chi_i = \frac{L}{r}\frac{2\bar\phi'}{g^2\bar\phi^3}\eta_i
 +\frac{1}{r^3}\partial_rr^3\frac{2\bar\phi'}{g^2\bar\phi^3}\zeta_i
 -\xi\zeta_i,                                                            \\
  &
 \left(
  \Delta_J-2\frac{\bar\phi'}{\bar\phi}\frac{1}{r^2}\partial_rr^2-g^2\bar\phi^2
 \right)\eta_i
 = -2\frac{L}{r}\frac{\bar\phi'}{\bar\phi}\zeta_i,                       \\
  & \Delta_J\zeta_i = 0.
\end{align}
There are useful relations as
\begin{align}
 \frac{1}{r^3}\partial_rr^3\Psi^{\rm (top)}_i
  & = \frac{L}{r}\Psi_i^{\rm (mid)}-\xi\zeta_i,
 \label{eq_relationPositiveJ1}                  \\
 \frac{1}{r}\partial_rr\Psi^{\rm (mid)}_i
  & = \frac{L}{r}\Psi^{\rm (top)}_i+\eta_i,
 \label{eq_relationPositiveJ2}                  \\
 \Psi^{\rm (bot)}_i
  & = \frac{rg\bar\phi}{L}\Psi^{\rm (mid)}_i
 -\frac{1}{g\bar\phi L}\frac{1}{r}\partial_rr^2\eta_i
 +\frac{1}{g\bar\phi}\zeta.
 \label{eq_relationPositiveJ3}
\end{align}

The first solution is obtained by setting $\zeta_1 = 0,\eta_1 = 0$, and
\begin{align}
 \chi_1 = r^{2J}.
\end{align}
We get
\begin{align}
 \Psi_1 =
 \begin{pmatrix}
  2Jr^{2J-1} \\
  Lr^{2J-1}  \\
  g\bar\phi r^{2J}
 \end{pmatrix}.
\end{align}

For the second solution, we set $\zeta_2 = 0$ and
\begin{align}
 \eta_2 = f^{(\eta)},
\end{align}
where
\begin{align}
 f^{(\eta)} = r^{2J}
 \left(
  1+\frac{|\lambda|}{8}\bar\phi_C^2r^2
 \right)^{z_g}{}_2F_1
 \left(
  1+z_g,2(J+1)+z_g,2(J+1),-\frac{|\lambda|}{8}\bar\phi_C^2r^2
 \right),
\end{align}
with
\begin{align}
 z_g = -\frac{1}{2}\left(
  1-\sqrt{1-8\frac{g^2}{|\lambda|}}
 \right).
\end{align}
It satisfies
\begin{align}
 \left(
  \Delta_J-2\frac{\bar\phi'}{\bar\phi}\frac{1}{r^2}\partial_rr^2-g^2\bar\phi^2
 \right)f^{(\eta)} = 0.
\end{align}
Its asymptotic behavior is given by
\begin{align}
 \lim_{r\to0}\frac{f^{(\eta)}(r)}{r^{2J}}
  & = 1,                              \\
 \lim_{r\to\infty}\frac{f^{(\eta)}(r)}{r^{2J-2}}
  & = \frac{8}{|\lambda|\bar\phi_C^2}
 \frac{\Gamma(2J+1)\Gamma(2J+2)}{\Gamma(2J+1-z_g)\Gamma(2J+2+z_g)}.
\end{align}
Then, using eqs.~\eqref{eq_relationPositiveJ1}, \eqref{eq_relationPositiveJ2}, and
\eqref{eq_relationPositiveJ3}, we have
\begin{align}
 \lim_{r\to0}\Psi_2      & =
 \begin{pmatrix}
  \frac{Lr^{2J+1}}{8(J+1)}   \\
  \frac{J+2}{4(J+1)}r^{2J+1} \\
  -\frac{2}{g\bar\phi_C}\frac{J+1}{L}r^{2J}
 \end{pmatrix}, \\
 \lim_{r\to\infty}\Psi_2 & =
 \begin{pmatrix}
  \frac{4L}{(2J+1)|\lambda|\bar\phi^2}r^{2J-1}\ln r       \\
  \frac{8(J+1)}{(2J+1)|\lambda|\bar\phi_C^2}r^{2J-1}\ln r \\
  -\frac{2J}{gL\bar\phi_C}r^{2J}
 \end{pmatrix}\times\frac{\Gamma(2J+1)\Gamma(2J+2)}{\Gamma(2J+1-z_g)\Gamma(2J+2+z_g)}.
\end{align}

The last solution can be obtained with
\begin{align}
 \zeta_3 = r^{2J}.
\end{align}
The asymptotic form of $\eta_3$ is given by
\begin{align}
 \lim_{r\to0}\frac{\eta_3(r)}{r^{2J+2}} = \frac{L|\lambda|\bar\phi_C^2}{16(J+1)}, \\
 \lim_{r\to\infty}\frac{\eta_3(r)}{r^{2J}} = \frac{L}{2(J+1)}.
\end{align}
Using eqs.~\eqref{eq_relationPositiveJ1}, \eqref{eq_relationPositiveJ2},
and \eqref{eq_relationPositiveJ3}, we have
\begin{align}
 \lim_{r\to0}\Psi_3      & =
 \begin{pmatrix}
  -\frac{\xi}{4}r^{2J+1}   \\
  -\frac{J\xi}{2L}r^{2J+1} \\
  \frac{1}{g\bar\phi_C}r^{2J}
 \end{pmatrix}, \\
 \lim_{r\to\infty}\Psi_3 & =
 \begin{pmatrix}
  \frac{J-\xi(J+1)}{4(J+1)}r^{2J+1}         \\
  \frac{J[(J+2)-\xi(J+1)]}{2L(J+1)}r^{2J+1} \\
  \frac{g}{|\lambda|\bar\phi_C}\frac{(J+2)-\xi(J+1)}{(J+1)^2}r^{2J}
 \end{pmatrix}.
\end{align}

As for the solutions around the false vacuum, we have
\begin{align}
 (\hat\Psi_1~\hat\Psi_2~\hat\Psi_3)=
 \begin{pmatrix}
  2Jr^{2J-1} & \frac{(J+1)\xi-J}{2L^2}r^{2J+1}        & 0      \\
  Lr^{2J-1}  & \frac{(J+1)\xi-(J+2)}{4L(J+1)}r^{2J+1} & 0      \\
  0          & 0                                      & r^{2J}
 \end{pmatrix}.
\end{align}

Using them, we obtain
\begin{align}
 \frac{\det\mathcal M^{(S,L,\varphi)}_J}{\det\widehat{\mathcal M}^{(S,L,\varphi)}_J}
  & =
 \left(
  \lim_{r\to\infty}\frac{\det[\Psi_1~\Psi_2~\Psi_3]}{\det[\hat\Psi_1~\hat\Psi_2~\hat\Psi_3]}
 \right)
 \left(
  \lim_{r\to0}\frac{\det[\Psi_1~\Psi_2~\Psi_3]}{\det[\hat\Psi_1~\hat\Psi_2~\hat\Psi_3]}
 \right)^{-1}\nonumber                                                                    \\
  & = \frac{J}{J+1}\frac{\Gamma(2J+1)\Gamma(2J+2)}{\Gamma(2J+1-z_g)\Gamma(2J+2+z_g)}.
\end{align}

Meanwhile, the functional determinant for $(T)$ modes can be obtained by
replacing $\kappa\to g^2$ in eq.~\eqref{eq_scalarDeterminant}. We have
\begin{align}
 \frac{\det\mathcal M_J^{(T)}}{\det\widehat{\mathcal M}_J^{(T)}}
 = \frac{\Gamma(2J+1)\Gamma(2J+2)}{\Gamma(2J+1-z_g)\Gamma(2J+2+z_g)}.
\end{align}

For $J = 0$, we have a similar decomposition of solutions
\cite{Endo:2017gal,Endo:2017tsz}, which is given by
\begin{align}
 \Psi_i\equiv
 \begin{pmatrix}
  \Psi_i^{\rm (top)} \\
  \Psi_i^{\rm (bot)}
 \end{pmatrix} =
 \begin{pmatrix}
  \partial_r\chi_i \\
  g\bar\phi\chi_i
 \end{pmatrix}
 +
 \begin{pmatrix}
  -\frac{2\bar\phi'}{g^2\bar\phi^3}\zeta_i \\
  \frac{1}{g\bar\phi}\zeta_i
 \end{pmatrix},
\end{align}
where
\begin{align}
  & \Delta_0\chi_i =
 \frac{1}{r^3}\partial_rr^3\frac{2\bar\phi'}{g^2\bar\phi^3}\zeta_i-\xi\zeta_i, \\
  & \Delta_0\zeta_i =
 g\bar\phi \Xi.
 \label{eq_gauge0zeta}
\end{align}
Here, $\Xi$ is a source term, which will be used later. For the moment,
$\Xi = 0$.

There are useful relations;
\begin{align}
 \frac{1}{r^3}\partial_rr^3\Psi_i^{\rm (top)}
  & = -\xi\zeta_i,
 \label{eq_relationZero1}                                            \\
 \partial_r\frac{1}{g\bar\phi}\Psi_i^{\rm (bot)}
  & = \Psi_i^{\rm (top)}+\frac{1}{g^2\bar\phi^2}\partial_r\zeta_i.
 \label{eq_relationZero2}
\end{align}

The first solution to $\mathcal M_0^{(S,\varphi)}\Psi = 0$ is given by
\begin{align}
 \Psi_1 =
 \begin{pmatrix}
  0 \\
  \frac{\bar\phi}{\bar\phi_C}
 \end{pmatrix}.
 \label{eq_gauge0psi1}
\end{align}
Meanwhile, another solution can be obtained with
\begin{align}
 \zeta_2 = 1.
\end{align}
Integrating eqs.~\eqref{eq_relationZero1} and \eqref{eq_relationZero2},
we get
\begin{align}
 \Psi_2 =
 \begin{pmatrix}
  -\xi\frac{r}{4} \\
  -\xi g\bar\phi\frac{r^2}{8}
 \end{pmatrix}.
\end{align}

Using them, we obtain
\begin{align}
 \frac{\det\mathcal M^{(S,\varphi)}_0}{\det\widehat{\mathcal M}^{(S,\varphi)}_0}
  & = \left(
  \lim_{r\to\infty}\frac{\det[\Psi_1~\Psi_2]}{\det[\hat\Psi_1~\hat\Psi_2]}
 \right)
 \left(
  \lim_{r\to0}\frac{\det[\Psi_1~\Psi_2]}{\det[\hat\Psi_1~\hat\Psi_2]}
 \right)^{-1}\nonumber \\
  & = 0,
\end{align}
which is due to the existence of zero modes.

%
%as a result of the global symmetry \cite{Kusenko:1996bv},
%\begin{align}
%  \Phi\to e^{iT\theta}\Phi.
%\end{align}

In summary, we have
\begin{align}
 \ln\mathcal A^{(A_\mu,\varphi)}
 = -\frac{1}{2}\sum_{J=0}^{\infty}(2J+1)^2
 \ln\frac{\det\mathcal M^{(A_\mu,\varphi)}_J}
 {\det\widehat{\mathcal M}^{(A_\mu,\varphi)}_J},
 \label{eq_gaugeSLT}
\end{align}
with
\begin{align}
 \frac{\det\mathcal M^{(A_\mu,\varphi)}_J}{\det\widehat{\mathcal M}^{(A_\mu,\varphi)}_J}
 = \frac{J}{J+1}
 \left[
  \frac{\Gamma(2J+1)\Gamma(2J+2)}{\Gamma(2J+1-z_g)\Gamma(2J+2+z_g)}
 \right]^3.
\end{align}

%%%%%%%%%%%%%%%%%%%%
%Zero modes
\section{Zero Modes}
\label{apx_zeromode}
\setcounter{equation}{0}

Here, we discuss the zero modes associated with invariance under
dilatation, translation, and global transformation.

\subsection{Dilatational zero mode}

In $J = 0$ of the Higgs fluctuation, we have a dilatational zero
mode. The dilatational transformation is parametrized by $\bar\phi_C$
and is given by
\begin{align}
 h\to h+\bar\phi_C\tilde{\fnc}_DY_{0,0,0}+\mathcal O(\bar\phi_C^2).
\end{align}
where\footnote{Notice that $Y_{0,0,0} = 1/\sqrt{2\pi^2}$.}
\begin{align}
 \tilde{\fnc}_D(r) = \sqrt{2\pi^2}\frac{d\bar\phi}{d\bar\phi_C}.
\end{align}

We regulate the determinant as
\begin{align}
 \frac{\det[-\Delta_0-3|\lambda|\bar\phi^2+\nu \rho_D(r)]}{\det[-\Delta_0]},
\end{align}
where\footnote
%%
{A prescription used in \cite{Chigusa:2017dux} is consistent with our
  argument, where $\rho_D$ is a constant.}
%%
\begin{align}
 \rho_D(r) = \frac{15}{16\pi}|\lambda|^2\bar\phi_C^2\bar\phi(r)^2.
\end{align}
The ratio of the determinant can be obtained from
eq.~\eqref{eq_scalarDeterminant} with replacement
$\kappa\to-3|\lambda|+\frac{15}{16\pi}|\lambda|^2\bar\phi_C^2\nu$. Expanding
with respect to $\nu$, we have
\begin{align}
 \frac{\det[-\Delta_0-3|\lambda|\bar\phi^2+\nu \rho_D(r)]}{\det[-\Delta_0]}
 = -\nu\frac{|\lambda|}{16\pi}\bar\phi_C^2+\mathcal O(\nu^2).
\end{align}
Thus, we get
\begin{align}
 \left|
 \frac{\det[-\Delta_0-3|\lambda|\bar\phi^2]}{\det[-\Delta_0]}
 \right|^{-1/2}
 \to \int d\bar\phi_C\sqrt{\frac{16\pi}{|\lambda|}}\frac{1}{\bar\phi_C}.
\end{align}
Here, we take an absolute value since there is a negative mode
\cite{Coleman:1977py,Callan:1977pt}.

\subsection{Translational zero mode}

In $J = 1/2$ of the Higgs fluctuation, we have translational zero modes.
The translation is parametrized by the center of the bounce, $z_\mu$, as
\begin{align}
 h \to h+z_\mu \tilde{\fnc}_T Y_{1/2,\mu}+\mathcal O(z^2),
\end{align}
where
\begin{align}
  \tilde{\fnc}_T(r) & = -\frac{\pi}{\sqrt{2}}\frac{d\bar\phi}{dr}, \\
  Y_{1/2,\mu}(\hat{x}) & = \frac{\sqrt{2}}{\pi}\hat{x}_\mu,
\end{align}
with $\hat{x}_\mu = x_\mu/|x|$.  Notice that $Y_{1/2,\mu}$ is given by
a linear combination of $Y_{1/2,m_A,m_B}$ and that it has the same
normalization as $Y_{1/2,m_A,m_B}$.  We regularize the zero mode as
\begin{align}
 \frac{\det[-\Delta_{1/2}-3|\lambda|\bar\phi^2+\rho_T(r)\nu]}
 {\det[-\Delta_{1/2}]},
\end{align}
where
\begin{align}
 \rho_T(r) = \frac{3|\lambda|}{4\pi}
\end{align}
Then, we have
\begin{align}
 \left(
  \frac{\det[-\Delta_{1/2}-3|\lambda|\bar\phi^2]}{\det[-\Delta_{1/2}]}
 \right)^{-2}\to\int d^4z
 \lim_{r\rightarrow\infty}
 \left(
  \frac{\check f^{(h)}_{1/2}}{r}
 \right)^{-2},
\end{align}
where
\begin{align}
 [-\Delta_{1/2}-3|\lambda|\bar\phi^2]\check f^{(h)}_{1/2} = -\rho_Tf_{1/2}^{(h)}.
\end{align}
with
\begin{align}
 f_{1/2}^{(h)} = -\frac{4}{|\lambda|\bar\phi_C^3}\frac{d\bar\phi}{dr}.
\end{align}
The solution behaves as
\begin{align}
 \lim_{r\to\infty}\frac{\check f^{(h)}_{1/2}}{r} = \frac{1}{4\pi\bar\phi_C^2}.
\end{align}
Thus, we obtain
\begin{align}
 \left(
  \frac{\det[-\Delta_{1/2}-3|\lambda|\bar\phi^2]}{\det[-\Delta_{1/2}]}
 \right)^{-2}\to\int d^4z (4\pi\bar\phi_C^2)^2 =
 \left(
  \frac{32\pi}{|\lambda|}
 \right)^2
 \left(
  \frac{|\lambda|}{8}\bar\phi_C^2
 \right)^2\mathcal V_{\rm 4D},
\end{align}
where $\mathcal V_{\rm 4D}$ is the volume of spacetime.

\subsection{Gauge zero mode}

In the $J = 0$ contribution of the gauge field, we have a gauge zero
mode \cite{Kusenko:1996bv}.  For the case of the $U(1)$ gauge
symmetry, the bounce solution is parameterized as Eq.\
\eqref{???inSec2} with the parameter $\theta$.  The path integral over
the gauge zero mode can be understood as the integration over the
variable $\theta$.  

With the change of $\theta$, the NG mode transforms as 
\begin{align}
  \varphi \to 
  \varphi + \theta \tilde{\fnc}_G Y_{0,0,0} +\mathcal O(\theta^2),
\end{align}
where
\begin{align}
  \tilde{\fnc}_G = \sqrt{2\pi^2}J_G\bar\phi.
\end{align}
and $J_G$ is the Jacobian for the group space integral. For $U(1)$ case,
$J_G = 1$, with which the volume of the $U(1)$ group space is
\begin{align}
  {\cal V}_{U(1)} = \int_0^{2\pi} d\theta = 2\pi.
\end{align}
Using the equation of motion of the bounce solution, we can show
\begin{align}
  \mathcal M^{(S,\varphi)}_0 \tilde \fnc_G = 0.
\end{align}

\rem{TM: I will modify the following later.}

We regularize the zero mode as
\begin{align}
 \frac{\det
 \left[
  \mathcal M_0^{(S,\varphi)}+\nu\rho_G(r)
 \right]}{\det\widehat{\mathcal M}_0^{(S,\varphi)}},
\end{align}
with
\begin{align}
 \rho_G(r) = \frac{|\lambda|^2}{16\pi J_G^2}\bar\phi_C\bar\phi.
\end{align}

To evaluate it at the leading order in $\nu$, we evaluate\footnote{If we
 do a similar calculation for $\Psi_2$, we get
 $\lim_{r\to\infty}\det[\Psi_1~\check\Psi_2] = 0$. Thus, it is enough to
 calculate $\check\Psi_1$.}
\begin{align}
 \mathcal M^{(S,\varphi)}_0\check\Psi_1(r) = -\rho_G(r)\Psi_1(r).
\end{align}

The solution can be obtained by setting
\begin{align}
 \Xi(r) = \rho_G(r)\Psi_1^{\rm (bot)}(r),
\end{align}
in eq.~\eqref{eq_gauge0zeta}.
We can solve $\zeta_1$ as
\begin{align}
 \zeta_1(r) = -\frac{|\lambda|g}{16\pi J_G^2}\bar\phi(r).
\end{align}
Integrating eqs.~\eqref{eq_relationZero1} and \eqref{eq_relationZero2},
we get $\check\Psi_1$. Its asymptotic form is given by
\begin{align}
 \lim_{r\to\infty}\check\Psi_1 =
 \begin{pmatrix}
  \frac{\xi g}{4\pi J_G^2\bar\phi_C r} \\
  \frac{|\lambda|}{16\pi J_G^2}
 \end{pmatrix}.
\end{align}
Thus, we have
\begin{align}
  \left(
    \frac{\det\mathcal M^{(S,\varphi)}_0}{\det\widehat{\mathcal M}^{(S,\varphi)}_0}
  \right)^{-1/2}
  \to &\ 
  \int d\theta
  \left(
    \lim_{r\to\infty}\frac{\det[\check\Psi_1(r)~\Psi_2(r)]}{r}
  \right)^{-1/2}
  \left(
    \lim_{r\to0}\frac{\det[\Psi_1(r)~\Psi_2(r)]}{r}
  \right)^{1/2}
  \nonumber \\ & =
  \int d\theta J_G \sqrt{\frac{16\pi}{|\lambda|}}.
\end{align}

%%%%%%%%%%%%%%%%%%%%
%Infinite sum
\section{Infinite Sum over Angular Momentum}
\label{apx_infiniteSum}
\setcounter{equation}{0}

In this appendix, we execute various infinite sums appearing in the
calculation of functional determinants.

We first evaluate the following sum;
\begin{align}
 I_B(z) = \sum_{J=0}^\infty\frac{(2J+1)^2}{(1+\varepsilon)^{2J}}
 \ln\frac{\Gamma(2J+1)\Gamma(2J+2)}{\Gamma(2J+1-z)\Gamma(2J+2+z)}.
\end{align}
Here, $J$ is half-integer, and $z$ is a complex number satisfying
\begin{align}
 -2<\Re(z)<1.
 \label{eq_rez}
\end{align}
For $\varepsilon>0$, the sum converges since $1/(1+\varepsilon)^{2J}$
goes to zero much faster than power of $2J$.

First, we rewrite the log-gamma functions with integrals of digamma
functions as
\begin{align}
 I_B(z) = \sum_{J=0}^{\infty}\frac{(2J+1)^2}{(1+\varepsilon)^{2J}}
 \int_\infty^{2J+1}dx
 \left[
  \psi_{\Gamma}(x)-\psi_{\Gamma}(x-z)+\psi_{\Gamma}(x+1)-\psi_{\Gamma}(x+1+z)
 \right],
\end{align}
where $\psi_{\Gamma}(z)$ is the digamma function.

Then, we use the following relation;
\begin{align}
 \psi_{\Gamma}(x)-\psi_{\Gamma}(y) = \int_0^1\frac{u^{y-1}-u^{x-1}}{1-u}du,
\end{align}
which is valid for $\Re(x)>0$ and $\Re(y)>0$.  We get
\begin{align}
 I_B(z) = \sum_{J=0}^{\infty}\frac{(2J+1)^2}{(1+\varepsilon)^{2J}}
 \int_\infty^{2J+1}dx\int_0^1du\frac{u^{x-z-1}(1-u^z)(1-u^{z+1})}{1-u}.
\end{align}
Notice that this is verified only in region \eqref{eq_rez}.

Then, we interchange the two integrals\footnote{This is justified when
 \begin{align}
  \int_\infty^{2J+1}dx\int_0^1du
  \left|
  \frac{u^{x-z-1}(1-u^z)(1-u^{z+1})}{1-u}
  \right|<\infty.
 \end{align}}
and integrate over $x$ first. We get
\begin{align}
 I_B(z) = \sum_{J=0}^{\infty}\frac{(2J+1)^2}{(1+\varepsilon)^{2J}}
 \int_0^1du\frac{u^{2J}(1-u^z)(1-u^{1+z})}{u^z(1-u)\ln u}.
\end{align}
Notice that the integral over $u$ does not diverge.

Since we have regularized the sum, the final result should be
finite. Thus, we can take the sum first. We get
\begin{align}
 I_B(z) = (1+\varepsilon)^2\int_0^1du
 \frac{(1-u^z)(1-u^{1+z})(1+u+\varepsilon)}{u^z(1-u)(1+\varepsilon-u)^3\ln u}.
\end{align}
We can see that new poles appear at $u = 1+\varepsilon$, but the
integral is still convergent for positive $\varepsilon$.

In order to execute the integral over $u$, we use the following
relation;
\begin{align}
 \frac{1-u^z}{\ln u} = -z\int_0^1dtu^{zt}.
\end{align}
We get
\begin{align}
 I_B(z) = -z(1+\varepsilon)^2\int_0^1du\int_0^1dt
 \frac{(1-u^{1+z})(1+u+\varepsilon)}{u^{z(1-t)}(1-u)(1+\varepsilon-u)^3}.
\end{align}

Interchanging the integrals and integrating over $u$, we get
\begin{eqnarray}
 I_B(z)& = &z\int_0^1dt
 \left[
  \frac{(1+\varepsilon)(z-1+tz\varepsilon-\varepsilon)}{\varepsilon^2}
  +\frac{(1+\varepsilon)^2(2+\varepsilon)[\psi_{\Gamma}(1+tz-z)
     -\psi_{\Gamma}(1+tz)]}{\varepsilon^3}\right.\nonumber\\
  &&\hspace{10ex}+\frac{2+(3+2z-2tz)\varepsilon
   +(1+z-tz)^2\varepsilon^2}{(1+tz-z)\varepsilon^3}{}_2F_1
  \left(
   1,1+tz-z,2+tz-z,\frac{1}{1+\varepsilon}
  \right)\nonumber\\
  &&\hspace{10ex}\left.-\frac{(1+\varepsilon)
   (2+\varepsilon-2tz\varepsilon+t^2z^2\varepsilon^2)}{(1+tz)\varepsilon^3}{}_2F_1
  \left(
   1,1+tz,2+tz,\frac{1}{1+\varepsilon}
  \right)
 \right],
\end{eqnarray}
where ${}_2F_1(a,b,c,z)$ is the hypergeometric function.

Since we do not need higher order terms in $\varepsilon$, we expand it
as
\begin{eqnarray}
 I_B(z)& = &\int_0^1dt
 \left\{
  -\frac{1}{\varepsilon^2}z(z+1)-\frac{1}{2\varepsilon}z(z+1)(4+z-2tz)
  +\frac{\ln\varepsilon}{6}z^2(1+z)(1+2z-6tz+6t^2z)\right.\nonumber\\
  &&\hspace{10ex}-z+z^2
  \left[
   -\frac{53}{36}+t+\frac{1}{6}(1-t)H(tz-z)+\frac{1}{6}tH(tz)
  \right]\nonumber\\
  &&\hspace{10ex}+\frac{z^3}{12}[-7+14t+6(1-t)^2H(tz-z)+6t^2H(tz)]\nonumber\\
  &&\hspace{10ex}\left.+\frac{z^4}{9}[-1+3(1-t)t+3(1-t)^3H(tz-z)+3t^3H(tz)]
  +\mathcal O(\varepsilon)
 \right\},
\end{eqnarray}
where $H(z)$ is the harmonic number.

Executing the final integral, we obtain
\begin{eqnarray}
 I_B(z)& = &-\frac{1}{\varepsilon^2}z(z+1)
 -\frac{2}{\varepsilon}z(z+1)
 +\frac{1}{6}z^2(z+1)^2\ln\varepsilon
 +\mathcal S_B(z)+\mathcal{O}(\epsilon),
\end{eqnarray}
where
\begin{eqnarray}
 \mathcal S_B(z)& = &\frac{1}{6}z(1+z)(1+2z)
 \left[
  \ln\Gamma(1+z)-\ln\Gamma(1-z)
 \right]\nonumber\\
 &&-
 \left(
  z+z^2+\frac{1}{6}
 \right)
 \left[
  \psi_{\Gamma}^{(-2)}(1+z)+\psi_{\Gamma}^{(-2)}(1-z)
 \right]\nonumber\\
 &&+(1+2z)
 \left[
  \psi_{\Gamma}^{(-3)}(1+z)-\psi_{\Gamma}^{(-3)}(1-z)
 \right]\nonumber\\
 &&-2
 \left[
  \psi_{\Gamma}^{(-4)}(1+z)+\psi_{\Gamma}^{(-4)}(1-z)
 \right]\nonumber\\
 &&+\frac{1}{6}\gamma_Ez^2(z+1)^2-z-\frac{35}{36}z^2-\frac{z^4}{18}\nonumber\\
 &&+\frac{1}{2}\ln 2\pi+2\ln A_G+\frac{\zeta(3)}{2\pi^2}.
\end{eqnarray}
Here, $A_G$ is the Glaisher number.

We can repeat the same procedure for the fermion case, where the
summation is given by
\begin{align}
 I_F(z) = \sum_{J=0}^\infty\frac{(2J+1)(2J+2)}{(1+\varepsilon)^{2J}}\ln
 \left[
  \frac{[\Gamma(2J+2)]^2}{\Gamma(2J+2+z)\Gamma(2J+2-z)}
 \right]^2,
\end{align}
for $-2<\Re(z)<2$.

The result is
\begin{eqnarray}
 I_F(z)& = &-\frac{2}{\varepsilon^2}z^2-\frac{5}{\varepsilon}z^2
 +\frac{1}{3}z^2(z^2-2)\ln\varepsilon+\mathcal S_F(z)+\mathcal{O}(\epsilon).
\end{eqnarray}
\begin{eqnarray}
 \mathcal S_F(z)& = &\frac{2}{3}z(z^2-1)
 \left[
  \ln\Gamma(1+z)-\ln\Gamma(1-z)
 \right]\nonumber\\
 &&-2
 \left(
  z^2-\frac{1}{3}
 \right)\left[\psi_{\Gamma}^{(-2)}(1+z)+\psi_{\Gamma}^{(-2)}(1-z)
 \right]\nonumber\\
 &&+4z[\psi_{\Gamma}^{(-3)}(1+z)-\psi_{\Gamma}^{(-3)}(1-z)]\nonumber\\
 &&-4[\psi_{\Gamma}^{(-4)}(1+z)+\psi_{\Gamma}^{(-4)}(1-z)]\nonumber\\
 &&+\frac{1}{3}z^2(z^2-2)\gamma_E-\frac{z^2}{9}(z^2+31)\nonumber\\
 &&+4\ln A_G+\frac{\zeta(3)}{\pi^2}.
\end{eqnarray}

Next, we consider
\begin{align}
 I_h = \sum_{J=1}^{\infty}\frac{(2J+1)^2}{(1+\varepsilon)^{2J}}
 \ln\frac{2J(2J-1)}{(2J+3)(2J+2)},
\end{align}
which is similar to $I_B(z)$, but $z$ is outside of the valid region.

Using a similar technique, we have
\begin{align}
 I_h & = -\sum_{J=1}^{\infty}\frac{(2J+1)^2}{(1+\varepsilon)^{2J}}\int_0^3dx
 \left[
  \frac{1}{x+2J}+\frac{1}{x+2J-1}
 \right]\nonumber                                                            \\
     & = -\frac{6}{\varepsilon^2}-\frac{12}{\varepsilon}+6\ln\varepsilon
 +\frac{3}{2}+6\gamma_E+12\ln A_G+9\ln2+5\ln3+\mathcal O(\varepsilon)
\end{align}
Here, we executed the sum first.

Finally, we consider
\begin{align}
 I_G = \sum_{J=1/2}^{\infty}\frac{(2J+1)^2}{(1+\varepsilon)^{2J}}\ln\frac{2J}{2J+2},
\end{align}
which appears in the gauge sector.

Repeating the same procedure, we have
\begin{align}
 I_G & = -\sum_{J=1/2}^{\infty}
 \frac{(2J+1)^2}{(1+\varepsilon)^{2J}}\int_0^2dx\frac{1}{x+2J}\nonumber           \\
     & = -\frac{2}{\varepsilon^2}-\frac{4}{\varepsilon}+\frac{2}{3}\ln\varepsilon
 -\frac{3}{2}+\frac{2}{3}\gamma_E+\ln2+4\ln A_G+\mathcal{O}(\epsilon).
\end{align}

%%%%%%%%%%%%%%%%%%%%
%MS-bar
\section{Renormalization with $\msbar$-scheme}
\label{apx_divpart}
\setcounter{equation}{0}

In this appendix, we relate $\eps{X}$ to $\epsD$.

\subsection{Scalar}

Let us start with the scalar contribution. We first calculate
$\left[\ln\mathcal A^{(\sigma)}\right]_{{\rm div}}$ with using
$\eps{\sigma}$ as a regulator. The expansion of
eq.~\eqref{eq_expansionEachJ} is exactly the same as that with respect
to $\kappa$. Thus, we have
\begin{align}
 \left[
  \ln\frac{\det[-\Delta_J+\kappa\bar\phi^2]}{\det[-\Delta_J]}
 \right]_{\mathcal O(\kappa^2)}
 =\frac{2\kappa}{|\lambda|}\frac{1}{2J+1}+
 \left(
  \frac{2\kappa}{|\lambda|}
 \right)^2
 \left(
  \frac{1}{2(2J+1)^2}+\frac{1}{2J+1}-\psi^{(1)}(2J+1)
 \right),
\end{align}
where $\psi^{(n)}(z)$ is the polygamma function.

Summing over $J$, we get
\begin{align}
 \left[
  \ln\mathcal A^{(\sigma)}
 \right]_{{\rm div},\eps{\sigma}}
  & = -\frac{1}{2}\sum_{J=0}^{\infty}\frac{(2J+1)^2}{(1+\eps{\sigma})^{2J}}
 \left[
  \ln\frac{\det[-\Delta_J+\kappa\bar\phi^2]}{\det[-\Delta_J]}
 \right]_{\mathcal O(\kappa^2)}\nonumber                                    \\
  & = -\frac{1}{2}
 \left[
  \frac{2\kappa}{|\lambda|}
  \left(
   \frac{1}{\eps{\sigma}^2}+\frac{2}{\eps{\sigma}}+\frac{\kappa}{3|\lambda|}\ln\eps{\sigma}
  \right)+\frac{2\kappa}{|\lambda|}
  \left(
   1+\frac{\kappa}{18|\lambda|}
  \right)
 \right]+\mathcal O(\eps{\sigma}).
\end{align}
As is expected, it has the same divergences as $\left[\ln\mathcal
 A^{(\sigma)}\right]_{\eps{\sigma}}$ has.

Next, we directly calculate $\left[\ln\mathcal A^{(\sigma)}\right]_{\rm
div}$ by using the dimensional regularization. We have
\begin{align}
 \left[
  \ln\mathcal A^{(\sigma)}
 \right]_{{\rm div},\eps{\sigma}}
  & = -\frac{1}{2}
 \left[
  \tr
  \left(
   \frac{1}{-\partial^2}\kappa\bar\phi^2
  \right)-\frac{1}{2}\tr
  \left(
   \frac{1}{-\partial^2}\kappa\bar\phi^2\frac{1}{-\partial^2}\kappa\bar\phi^2
  \right)
 \right]\nonumber                                                              \\
  & = -\frac{1}{2}
 \left[
  \kappa\int\frac{d^nk}{(2\pi)^n}\frac{1}{k^2}F[\bar\phi^2](0)\right.\nonumber \\
  & \hspace{7ex}\left.-\frac{\kappa^2}{2}
  \int\frac{d^nk}{(2\pi)^n}\frac{d^nk'}{(2\pi)^n}
  \frac{1}{k^2}F[\bar\phi^2](k-k')\frac{1}{k'^2}F[\bar\phi^2](k'-k)
 \right]\nonumber                                                              \\
  & = -\frac{1}{2}
 \left[
  -\frac{2\kappa^2}{3|\lambda|^2}
  \left(
   \frac{1}{2\epsD}+\frac{5}{6}+\gamma_E+\ln\frac{\mu R}{2}
  \right)
 \right],
\end{align}
where $F[\cdots]$ is the Fourier transform of the argument,
$n=4-2\varepsilon_D$, and
\begin{align}
 R = \sqrt{\frac{8}{|\lambda|}}\frac{1}{\bar\phi_C}.
\end{align}
We get a relation between the two regulators as
\begin{align}
 \left(
  \frac{1}{\eps{\sigma}^2}+\frac{2}{\eps{\sigma}}
  +\frac{\kappa}{3|\lambda|}\ln\eps{\sigma}\right)
 \to -1-\frac{\kappa}{3|\lambda|}
 \left[
  \frac{1}{2\epsD}+1+\gamma_E+\ln\frac{\mu R}{2}
 \right].
 \label{eq_ScalarRegulatorRel}
\end{align}

\subsection{Higgs}

The relation for the Higgs contribution can be obtained from
eq.~\eqref{eq_ScalarRegulatorRel} with replacement
$\kappa\to-3|\lambda|$;
\begin{align}
 \frac{1}{\eps{h}^2}+\frac{2}{\eps{h}}
 -\ln\eps{h}\to\frac{1}{2\epsD}+\gamma_E+\ln\frac{\mu R}{2}.
\end{align}
Notice that the zero modes have nothing to do with the divergence.

\subsection{Fermion}

Next, we obtain the relation for the fermion contribution. As discussed
in \cite{Andreassen:2017rzq}, it is better to expand with respect to
$y$. The expansion in eq.~\eqref{eq_deltaMExpansion} is equivalent to
\begin{align}
 \left[\ln\mathcal A^{(\psi)}\right]_{{\rm div},\eps{\psi}}
  & = \sum_{J=0}^{\infty} \frac{(2J+1)(2J+2)}{(1+\eps{\psi})^{2J}}\nonumber \\
  & \hspace{3ex}
 \left\{
  \left[
   \ln\frac{\det\mathcal M^{(\psi)}_J}{\det\widehat{\mathcal M}^{(\psi)}_J}
  \right]_{\mathcal O(y^2)}+
  \left[
   \ln\frac{\det\mathcal M^{(\psi)}_{J,{\rm diag}}}{\det\widehat{\mathcal M}^{(\psi)}_{J,{\rm diag}}}
  \right]_{\mathcal O(y^4)}-
  \left[
   \ln\frac{\det\mathcal M^{(\psi)}_{J,{\rm diag}}}{\det\widehat{\mathcal M}^{(\psi)}_{J,{\rm diag}}}
  \right]_{\mathcal O(y^2)}
 \right\},
\end{align}
where
\begin{align}
 \mathcal M^{(\psi)}_{J,{\rm diag}} =
 \begin{pmatrix}
  -\Delta_J+\frac{y^2}{2}\bar\phi^2 & 0                                       \\
  0                                 & -\Delta_{J+1/2}+\frac{y^2}{2}\bar\phi^2
 \end{pmatrix},
\end{align}
and the hatted operator is obtained by replacement $\bar\phi\to0$.  From
eq.~\eqref{eq_scalarDeterminant}, we have
\begin{align}
 \frac{\det\mathcal M^{(\psi)}_{J,{\rm diag}}}{\det\widehat{\mathcal M}^{(\psi)}_{J,{\rm diag}}}
 = \frac{\Gamma(2J+1)\Gamma(2J+2)\Gamma(2J+2)\Gamma(2J+3)}
 {\Gamma(2J+1-\tilde z_y)\Gamma(2J+2+\tilde z_y)\Gamma(2J+2-\tilde z_y)\Gamma(2J+3+\tilde z_y)},
\end{align}
where
\begin{align}
 \tilde{z}_y = -\frac{1}{2}\left(1-\sqrt{1-4\frac{y^2}{|\lambda|}}\right).
\end{align}
Together with eq.~\eqref{eq_FermionDeterminant}, we have
\begin{align}
 \left[\ln\mathcal A^{(\psi)}\right]_{{\rm div},\eps{\psi}}
  & = \sum_{J=0}^{\infty}
 \frac{(2J+1)(2J+2)}{(1+\eps{\psi})^{2J}}
 \left[
  \frac{2y^2}{|\lambda|}\psi_{\Gamma}^{(1)}(2J+2)\right.\nonumber \\
  & \hspace{3ex}-\frac{y^4}{|\lambda|^2}
  \left(
   \psi_{\Gamma}^{(0)}(2J+1)-\psi_{\Gamma}^{(0)}(2J+3)
  \right)\nonumber                                                \\
  & \hspace{3ex}\left.-\frac{y^4}{2|\lambda|^2}
  \left(
   \psi_{\Gamma}^{(1)}(2J+1)+2\psi_{\Gamma}^{(1)}(2J+2)+\psi_{\Gamma}^{(1)}(2J+3)
  \right)
 \right]\nonumber                                                 \\
  & = \frac{y^2}{|\lambda|}
 \left(
  \frac{2}{\eps{\psi}^2}+\frac{5}{\eps{\psi}}+\frac{1}{3}\frac{y^2+2|\lambda|}{|\lambda|}\ln\eps{\psi}
 \right)+\frac{31}{9}\frac{y^2}{|\lambda|}+\frac{1}{18}\frac{y^4}{|\lambda|^2}+\mathcal O(\eps{\psi}).
\end{align}
Meanwhile, we can calculate it by using the dimensional regularization;
\begin{align}
 \left[\ln\mathcal A^{(\psi)}\right]_{{\rm div},\epsD}
  & = y^2\tr
 \left[
  \frac{1}{-\partial^2}\bar\phi^2
 \right]-\frac{y^4}{4}\tr
 \left[
  \frac{1}{-\partial^2}\bar\phi^2\frac{1}{-\partial^2}\bar\phi^2
 \right]-\frac{y^2}{2}\tr
 \left[
  \frac{1}{-\partial^2}(\partial_\mu\bar\phi)\frac{1}{-\partial^2}(\partial_\mu\bar\phi)
 \right]\nonumber               \\
  & = -\frac{y^4}{3|\lambda|^2}
 \left(
  \frac{1}{2\epsD}+\frac{5}{6}+\gamma_E+\ln\frac{\mu R}{2}
 \right)-\frac{2y^2}{3|\lambda|}
 \left(
  \frac{1}{2\epsD}+\frac{13}{12}+\gamma_E+\ln\frac{\mu R}{2}
 \right).
\end{align}
Thus, we have
\begin{align}
  &
 \left(
  \frac{2}{\eps{\psi}^2}+\frac{5}{\eps{\psi}}+\frac{1}{3}\frac{y^2+2|\lambda|}{|\lambda|}\ln\eps{\psi}
 \right)\nonumber                            \\
  & \hspace{3ex} \to -\frac{y^2}{3|\lambda|}
 \left(
  \frac{1}{2\epsD}+1+\gamma_E+\ln\frac{\mu R}{2}
 \right)-\frac{2}{3}
 \left(
  \frac{1}{2\epsD}+\frac{25}{4}+\gamma_E+\ln\frac{\mu R}{2}
 \right).
\end{align}

\subsection{Gauge field}

Finally, we obtain the relation for the gauge contribution.  Since the
fluctuation operator \eqref{eq_gaugeFluct} has terms proportional to
$\partial_\mu$, the expansion given in eq.~\eqref{eq_deltaMExpansion}
is not enough to subtract all the divergences. In principle, one can
expand up to $\mathcal O(\delta \mathcal M^4)$, but it is not so
efficient. Instead, we use a different gauge fixing function given by
\begin{align}
 \mathcal F_{\rm BG} = (\partial_\mu A_\mu-\xi g\bar\phi\varphi),
\end{align}
which we call the background gauge. We take $\xi = 1$ in the
following, but we can show that the result does not depend on $\xi$
\cite{????}.

Using the same basis given in eq.~\eqref{eq_gaugeBasis}, we have
\begin{align}
 \ln\mathcal A_{\rm BG}^{(A_\mu,\varphi)}
  & = -\frac{1}{2}
 \ln\frac{\det\mathcal M^{(A_\mu,\varphi)}_{\rm BG}}
 {\det\widehat{\mathcal M}^{(A_\mu,\varphi)}_{\rm BG}} \\
  & = -\frac{1}{2}
 \ln\frac{\det\mathcal M^{(S,\varphi)}_{0,\rm BG}}
 {\det\widehat{\mathcal M}^{(S,\varphi)}_{0,\rm BG}}
 -\frac{1}{2}\sum_{J=1/2}^{\infty}(2J+1)^2
 \left[
  \ln\frac{\det\mathcal M^{(S,L,\varphi)}_{J,\rm BG}}
  {\det\widehat{\mathcal M}^{(S,L,\varphi)}_{J,\rm BG}}
  +2\ln\frac{\det\mathcal M^{(T)}_{J,\rm BG}}
  {\det\widehat{\mathcal M}^{(T)}_{J,\rm BG}}
 \right],
\end{align}
and we also have Faddeev-Popov contributions,
\begin{align}
 \ln\mathcal A_{\rm BG}^{(c,\bar{c})}
  & = \ln\frac{\det\mathcal M^{(c,\bar{c})}_{\rm BG}}
 {\det\widehat{\mathcal M}^{(c,\bar{c})}_{\rm BG}}    \\
  & = \sum_{J=0}^{\infty}(2J+1)^2
 \ln\frac{\det\mathcal M^{(c,\bar{c})}_{J,\rm BG}}
 {\det\widehat{\mathcal M}^{(c,\bar{c})}_{J,\rm BG}}.
\end{align}
Here,
\begin{align}
 \mathcal M^{(c,\bar{c})}_{\rm BG} & = -\partial^2+g^2\bar\phi^2,
\end{align}
\begin{align}
 \mathcal M^{(A_\mu,\varphi)}_{\rm BG} =
 \begin{pmatrix}
  -\partial^2\delta_{\mu\nu}+g^2\bar\phi^2 & 2g(\partial_\nu\bar\phi)          \\
  2g(\partial_\mu\bar\phi)                 & -\partial^2+m_{a}^2+g^2\bar\phi^2
 \end{pmatrix}.
\end{align}
For the partial waves, we have
\begin{align}
 \mathcal M^{(T)}_{J,\rm BG} & = \mathcal M^{(c,\bar{c})}_{J,\rm BG} = \mathcal M^{(T)}_J,
\end{align}
\begin{align}
 \mathcal M^{(S,L,\varphi)}_{J,\rm BG} & =
 \begin{pmatrix}
  -\Delta_J+\frac{3}{r^2}+g^2\bar\phi^2 & -\frac{2L}{r^2}                         & 2g\bar\phi'                   \\
  -\frac{2L}{r^2}                         & -\Delta_J-\frac{1}{r^2}+g^2\bar\phi^2 & 0                               \\
  2g\bar\phi'                           & 0                                       & -\Delta_J+m_a^2+g^2\bar\phi^2
 \end{pmatrix},
\end{align}
and
\begin{align}
 \mathcal M^{(S,\varphi)}_{0,\rm BG} & =
 \begin{pmatrix}
  -\Delta_{1/2}+g^2\bar\phi^2 & 2g\bar\phi'                   \\
  2g\bar\phi'                 & -\Delta_0+m_a^2+g^2\bar\phi^2
 \end{pmatrix}.
\end{align}
The hatted operators are defined through replacements $\bar\phi\to0$ and
$\bar\phi'\to0$.

A prominent feature of this gauge fixing is that we can take a basis so
that $\widehat{\mathcal M}^{(S,L,\varphi)}_{J,\rm BG}$ is diagonalized;
\begin{align}
 B^t\mathcal M^{(S,L,\varphi)}_{J,\rm BG} B =
 \begin{pmatrix}
  -\Delta_{J-1/2}+g^2\bar\phi^2    & 0                                     & 2\sqrt{\frac{J}{2J+1}}g\bar\phi'    \\
  0                                  & -\Delta_{J+1/2}+g^2\bar\phi^2       & -2\sqrt{\frac{J+1}{2J+1}}g\bar\phi' \\
  2\sqrt{\frac{J}{2J+1}}g\bar\phi' & -2\sqrt{\frac{J+1}{2J+1}}g\bar\phi' & -\Delta_J+m_a^2+g^2\bar\phi^2
 \end{pmatrix},
\end{align}
where
\begin{align}
 B =
 \begin{pmatrix}
  \sqrt{\frac{J}{2J+1}}   & -\sqrt{\frac{J+1}{2J+1}} & 0 \\
  \sqrt{\frac{J+1}{2J+1}} & \sqrt{\frac{J}{2J+1}}    & 0 \\
  0                       & 0                        & 1
 \end{pmatrix}.
\end{align}
Furthermore, the contributions from $(c,\bar{c})$ and $(T)$ are
exactly canceled out for $J>0$.

Repeating the same procedure using solutions given in
\cite{Endo:2017gal}, we can show that
\begin{align}
 \ln\frac{\det\mathcal M^{(S,L,\varphi)}_{J,\rm BG}}{\det\widehat{\mathcal M}^{(S,L,\varphi)}_{J,\rm BG}}
 +2\ln\frac{\det\mathcal M^{(T)}_{J,\rm BG}}{\det\widehat{\mathcal M}^{(T)}_{J,\rm BG}}
 -2\ln\frac{\det\mathcal M^{(c,\bar{c})}_{J,\rm BG}}{\det\widehat{\mathcal M}^{(c,\bar{c})}_{J,\rm BG}}
 = \ln\frac{\det\mathcal M^{(S,L,\varphi)}_{J}}{\det\widehat{\mathcal M}^{(S,L,\varphi)}_{J}}
 +2\ln\frac{\det\mathcal M^{(T)}_{J}}{\det\widehat{\mathcal M}^{(T)}_{J}}.
\end{align}
In particular, the structure of the divergence is the same. It means
that
\begin{align}
 \lim_{\eps{A}\to0}
 \left\{
  \left[
   \ln\mathcal A'^{(A_\mu,\varphi)}
  \right]_{\eps{A}}-
  \left[
   \ln\mathcal A'^{(A_\mu,\varphi)}_{\rm BG}
  \right]_{\eps{A}}-
  \left[
   \ln\mathcal A_{\rm BG}^{(c,\bar{c})}
  \right]_{\eps{A}}
 \right\} = (\rm finite).
\end{align}
Furthermore, as discussed in \cite{Endo:2017tsz}, we can show that
\begin{align}
 \lim_{\epsD\to0}
 \left\{
  \left[
   \ln\mathcal A'^{(A_\mu,\varphi)}
  \right]_{\epsD}-
  \left[
   \ln\mathcal A'^{(A_\mu,\varphi)}_{\rm BG}
  \right]_{\epsD}-
  \left[
   \ln\mathcal A_{\rm BG}^{(c,\bar{c})}
  \right]_{\epsD}
 \right\} = (\rm finite).
\end{align}
Since the finite part cannot depend on the regularization, we have
\begin{align}
  &
 \left[
  \ln\mathcal A'^{(A_\mu,\varphi)}
 \right]_{\epsD}-
 \left[
  \ln\mathcal A^{(A_\mu,\varphi)}_{\rm BG}
 \right]_{{\rm div},\epsD}-
 \left[
  \ln\mathcal A_{\rm BG}^{(c,\bar{c})}
 \right]_{{\rm div},\epsD}\nonumber \\
  & =
 \left[
  \ln\mathcal A'^{(A_\mu,\varphi)}
 \right]_{\eps{A}}-
 \left[
  \ln\mathcal A^{(A_\mu,\varphi)}_{\rm BG}
 \right]_{{\rm div},\eps{A}}-
 \left[
  \ln\mathcal A_{\rm BG}^{(c,\bar{c})}
 \right]_{{\rm div},\eps{A}},
\end{align}
where $[\cdots]_{{\rm div},\eps{A}}$ is defined accordingly to
eq.~\eqref{eq_regDiv}.

Let us evaluate the divergent part with $\eps{A}$ regularization.
Before proceeding, we define the following quantities.
\begin{align}
 K_J^{\rm (diag)}(\delta)
  & = \left[
  \ln\frac{\det[-\Delta_J+\delta\bar\phi^2]}{\det[-\Delta_J]}
 \right]_{\mathcal O(\delta^2)}
 \\
  & = \frac{2\delta}{|\lambda|}\frac{1}{2J+1}+
 \left(
  \frac{2\delta}{|\lambda|}
 \right)^2
 \left(
  \frac{1}{2(2J+1)^2}+\frac{1}{2J+1}-\psi^{(1)}(2J+1)
 \right),
\end{align}
and
\begin{align}
 K_J^{\rm (off)}(\delta)
  & =
 \left[
  \ln\frac{\det
   \begin{pmatrix}
    -\Delta_J
     & \delta\bar\phi' \\
    \delta\bar\phi'
     & -\Delta_{J+1/2}
   \end{pmatrix}}{\det
   \begin{pmatrix}
    -\Delta_J
     & 0               \\
    0
     & -\Delta_{J+1/2}
   \end{pmatrix}}
 \right]_{\mathcal O(\delta^2)}\nonumber \\
  & =
 \left[
  \ln\frac{\det
   \begin{pmatrix}
    -\Delta_J+\delta^2\bar\phi^2
     & \delta\bar\phi'                    \\
    \delta\bar\phi'
     & -\Delta_{J+1/2}+\delta^2\bar\phi^2
   \end{pmatrix}}{\det
   \begin{pmatrix}
    -\Delta_J
     & 0               \\
    0
     & -\Delta_{J+1/2}
   \end{pmatrix}}
 \right]_{\mathcal O(\delta^2)}\nonumber \\
  & \hspace{3ex}-
 \left[
  \ln\frac{\det[-\Delta_J+\delta^2\bar\phi^2]}{\det[-\Delta_J]}
 \right]_{\mathcal O(\delta^2)}-
 \left[
  \ln\frac{\det[-\Delta_{J+1/2}+\delta^2\bar\phi^2]}{\det[-\Delta_{J+1/2}]}
 \right]_{\mathcal O(\delta^2)}\nonumber \\
  & = \frac{\delta^2}{|\lambda|}
 \left[
  4\psi_\Gamma^{(1)}(2J+2)-\frac{2}{2J+1}-\frac{1}{J+1}
 \right].
\end{align}
Then,
\begin{align}
  &
 \left[
  \ln\mathcal A^{(A_\mu,\varphi)}_{\rm BG}
 \right]_{{\rm div},\eps{A}}+
 \left[
  \ln\mathcal A_{\rm BG}^{(c,\bar{c})}
 \right]_{{\rm div},\eps{A}}\nonumber                                         \\
  & = -\frac{1}{2}
 \left\{
  \left[
   \ln\frac{\det \mathcal M^{(S,\varphi)}_{0,\rm BG}}{\det \widehat{\mathcal M}^{(S,\varphi)}_{0,\rm BG}}
  \right]_{\mathcal O(\delta \mathcal M^2)}-2
  \left[
   \ln\frac{\det \mathcal M^{(c,\bar{c})}_{0,\rm BG}}{\det \widehat{\mathcal M}^{(c,\bar{c})}_{0,\rm BG}}
  \right]_{\mathcal O(\delta \mathcal M^2)}\right.\nonumber                   \\
  & \hspace{7ex}\left.+\sum_{J=1/2}^{\infty}\frac{(2J+1)^2}{(1+\eps{A})^{2J}}
  \left[
   \ln\frac{\det \mathcal M^{(S,L,\varphi)}_{J,\rm BG}}{\det \widehat{\mathcal M}^{(S,L,\varphi)}_{J,\rm BG}}
  \right]_{\mathcal O(\delta \mathcal M^2)}
 \right\},
\end{align}
where
\begin{align}
 \left[
  \ln\frac{\det \mathcal M^{(S,\varphi)}_{0,\rm BG}}{\det \widehat{\mathcal M}^{(S,\varphi)}_{0,\rm BG}}
 \right]_{\mathcal O(\delta \mathcal M^2)}
  & = K_{1/2}^{\rm (diag)}(g^2)+K_{0}^{\rm (diag)}(g^2-|\lambda|)+K_0^{\rm (off)}(2g), \\
 \left[
  \ln\frac{\det \mathcal M^{(S,L,\varphi)}_{J,\rm BG}}{\det \widehat{\mathcal M}^{(S,L,\varphi)}_{J,\rm BG}}
 \right]_{\mathcal O(\delta \mathcal M^2)}
  & = K_{J+1/2}^{\rm (diag)}(g^2)+K_{J-1/2}^{\rm (diag)}(g^2)
 +K_{J}^{\rm (diag)}(g^2-|\lambda|)\nonumber                                               \\
  & \hspace{3ex}+K_J^{\rm (off)}\left(-2\sqrt{\frac{J+1}{2J+1}}g\right)
 +K_{J-1/2}^{\rm (off)}\left(2\sqrt{\frac{J}{2J+1}}g\right),                               \\
 \left[
  \ln\frac{\det \mathcal M^{(c,\bar{c})}_{0,\rm BG}}{\det \widehat{\mathcal M}^{(c,\bar{c})}_{0,\rm BG}}
 \right]_{\mathcal O(\delta \mathcal M^2)}
  & = K_{0}^{\rm (diag)}(g^2).
\end{align}
Notice that $m_a^2 = -|\lambda|\bar\phi^2$, and that the contributions
from $(c,\bar{c})$ and $(T)$ are canceled out for $J>0$.  Summing
over $J$, we get
\begin{align}
 \left[
  \ln\mathcal A^{(A_\mu,\varphi)}_{\rm BG}
 \right]_{{\rm div},\eps{A}}+
 \left[
  \ln\mathcal A_{\rm BG}^{(c,\bar{c})}
 \right]_{{\rm div},\eps{A}}
  & = -\frac{1}{2}
 \left[
  \left(
   \frac{6g^2}{|\lambda|}-2
  \right)
  \left(
   \frac{1}{\eps{A}^2}+\frac{2}{\eps{A}}
  \right)+
  \left(
   \frac{2}{3}+\frac{2g^4}{|\lambda|^2}
  \right)\ln\eps{A}\right.\nonumber                                                                 \\
  & \hspace{7ex}\left.-\frac{17}{9}-\frac{2g^2}{3|\lambda|}-\frac{g^4}{3|\lambda|^2}(59-6\pi^2)
 \right]+\mathcal O(\eps{A}).
\end{align}

On the other hand, we have the result with the dimensional
regularization,
\begin{align}
  & \left[
  \ln\mathcal A^{(A_\mu,\varphi)}_{\rm BG}
 \right]_{{\rm div},\epsD}+
 \left[
  \ln\mathcal A_{\rm BG}^{(c,\bar{c})}
 \right]_{{\rm div},\epsD}\nonumber \\
  & = -\frac{1}{2}
 \left[
  (3g^2-|\lambda|)\tr
  \left(
   \frac{1}{-\partial^2}\bar\phi^2
  \right)-\frac{1}{2}(3g^4-2g^2|\lambda|+|\lambda|^2)\tr
  \left(
   \frac{1}{-\partial^2}\bar\phi^2\frac{1}{-\partial^2}\bar\phi^2
  \right)\right.\nonumber           \\
  & \hspace{7ex}\left.-4g^2\tr
  \left(
   \frac{1}{-\partial^2}(\partial_\mu\bar\phi)\frac{1}{-\partial^2}(\partial_\mu\bar\phi)
  \right)
 \right]\nonumber                   \\
  & = -\frac{1}{2}
 \left[
  -\frac{4g^2}{3|\lambda|}-2
  \left(
   \frac{1}{3}+\frac{2g^2}{|\lambda|}+\frac{g^4}{|\lambda|^2}
  \right)
  \left(
   \frac{1}{2\epsD}+\frac{5}{6}+\gamma_E+\ln\frac{\mu R}{2}
  \right)
 \right].
\end{align}

Thus, we get
\begin{align}
  &
 \left(
  \frac{6g^2}{|\lambda|}-2
 \right)
 \left(
  \frac{1}{\eps{A}^2}+\frac{2}{\eps{A}}
 \right)+
 \left(
  \frac{2}{3}+\frac{2g^4}{|\lambda|^2}
 \right)\ln\eps{A}\nonumber \\
  & \hspace{3ex} \to -2
 \left(
  \frac{1}{3}+\frac{2g^2}{|\lambda|}+\frac{g^4}{|\lambda|^2}
 \right)
 \left(
  \frac{1}{2\epsD}+1+\gamma_E+\ln\frac{\mu R}{2}
 \right)+2+2\frac{g^4}{|\lambda|^2}(10-\pi^2).
\end{align}

%%%%%%%%%%%%%%%%%%%%
%Threshold corrections
\section{Threshold Corrections}
\label{apx_threshold}
\setcounter{equation}{0}

In this appendix, we summarize the one-loop threshold corrections for
the models with extra fermions.  All of the threshold corrections in our
analysis can be written as
\begin{align}
 g^{\rm (below)} = g + \frac{1}{16\pi^2} \Delta_g,
\end{align}
where $g^{\rm (below)}$ is a coupling constant used below the matching
scale, while $g$ is that used above the scale.  For the notational
simplicity, we will only show the function $\Delta$ for each coupling
constant.  The function $\Delta$ also contains the extra fermion mass
$M_{\rm ex}$ and the matching scale $\mu$, which is equal to $M_{\rm ex}$
in our analysis.

\begin{itemize}
 \item Vector-like quarks $Q_{\rm ex}, \overline{Q}_{\rm
       ex}, D_{\rm ex}, \overline{D}_{\rm ex}$
\end{itemize}
\begin{align}
 \Delta_{g_1} &= - \frac{1}{5} g_1^2 \log \frac{\mu^2}{M_{\rm ex}^2},\\
 \Delta_{g_2} &= - g_2^2 \log \frac{\mu^2}{M_{\rm ex}^2}, \\
 \Delta_{g_3} &= - g_3^2 \log \frac{\mu^2}{M_{\rm ex}^2}, \\
 \Delta_{y_t} &= - 6 y_t y_{\rm ex}^2 \left[ \frac{1}{2} \log
 \frac{\mu^2}{M_{\rm ex}^2} - \frac{1}{3} \right],\\
 \Delta_{y_b} &= - 6 y_b y_{\rm ex}^2 \left[ \frac{1}{2} \log
 \frac{\mu^2}{M_{\rm ex}^2} - \frac{1}{3} \right],\\
 \Delta_{y_\tau} &= - 6 y_\tau y_{\rm ex}^2 \left[ \frac{1}{2} \log
 \frac{\mu^2}{M_{\rm ex}^2} - \frac{1}{3} \right],\\
 \Delta_{\lambda} &= - 24 \lambda y_{\rm ex}^2 \left[ \frac{1}{2} \log
 \frac{\mu^2}{M_{\rm ex}^2} - \frac{1}{3} \right] + 12 y_{\rm ex}^4 \left[
 \frac{1}{2} \log \frac{\mu^2}{M_{\rm ex}^2} - \frac{4}{3} \right].
\end{align}

\begin{itemize}
 \item Vector-like leptons $L_{\rm ex}, \overline{L}_{\rm
       ex}, E_{\rm ex}, \overline{E}_{\rm ex}$
\end{itemize}
\begin{align}
 \Delta_{g_1} &= - \frac{3}{5} g_1^2 \log \frac{\mu^2}{M_{\rm ex}^2},\\
 \Delta_{g_2} &= - \frac{1}{3} g_2^2 \log \frac{\mu^2}{M_{\rm ex}^2}, \\
 \Delta_{g_3} &= 0, \\
 \Delta_{y_t} &= - 2 y_t y_{\rm ex}^2 \left[ \frac{1}{2} \log
 \frac{\mu^2}{M_{\rm ex}^2} - \frac{1}{3} \right],\\
 \Delta_{y_b} &= - 2 y_b y_{\rm ex}^2 \left[ \frac{1}{2} \log
 \frac{\mu^2}{M_{\rm ex}^2} - \frac{1}{3} \right],\\
 \Delta_{y_\tau} &= - 2 y_\tau y_{\rm ex}^2 \left[ \frac{1}{2} \log
 \frac{\mu^2}{M_{\rm ex}^2} - \frac{1}{3} \right],\\
 \Delta_{\lambda} &= - 8 \lambda y_{\rm ex}^2 \left[ \frac{1}{2} \log
 \frac{\mu^2}{M_{\rm ex}^2} - \frac{1}{3} \right] + 4 y_{\rm ex}^4 \left[
 \frac{1}{2} \log \frac{\mu^2}{M_{\rm ex}^2} - \frac{4}{3} \right].
\end{align}

\begin{itemize}
 \item Right-handed neutrino
\end{itemize}
\begin{align}
 \Delta_{g_i} &= 0\ (i = 1,2,3),\\
 \Delta_{y_t} &= - y_t y_{\rm ex}^2 \left[ \frac{1}{2} \log
 \frac{\mu^2}{M_{\rm ex}^2} + \frac{1}{4} \right],\\
 \Delta_{y_b} &= - y_b y_{\rm ex}^2 \left[ \frac{1}{2} \log
 \frac{\mu^2}{M_{\rm ex}^2} + \frac{1}{4} \right],\\
 \Delta_{y_\tau} &= y_\tau y_{\rm ex}^2 \left[ \frac{1}{4} \log
 \frac{\mu^2}{M_{\rm ex}^2} + \frac{3}{8} \right],\\
 \Delta_{\lambda} &= - 4 \lambda y_{\rm ex}^2 \left[ \frac{1}{2} \log
 \frac{\mu^2}{M_{\rm ex}^2} + \frac{1}{4} \right] + 2 y_{\rm ex}^4 \left[
 \frac{1}{2} \log \frac{\mu^2}{M_{\rm ex}^2} - \frac{1}{2} \right].
\end{align}

%%%%%%%%%%%%%%%%%%%%
%Numcerics
\section{Numerical Recipe}
\label{apx_recipe}
\setcounter{equation}{0}

In this appendix, we give fitting formulas of the prefactors at the
one-loop level.  Contrary to the analytic formulas including various
special functions with complex arguments, which may be inconvenient
for numerical calculations, the fitting formulas give a simple
procedure to perform a numerical calculations of the decay rate with
saving computatioal time.  Compared to the analytic expressions, the
errors of the fitting formulas are $0.05\%$ or better.

\rem{ I removed the expression of $\gamma$.  I believe that we do not
  have to take $\mu=1/R$ in the following expressions.}

% The decay rate is given by
% \begin{align}
%   \gamma = \int d\ln R \frac{1}{R^4}
%   \left[
%     \mathcal A'^{(h)}\mathcal A^{(\sigma)}
%     \mathcal A^{(\psi)}\mathcal A^{(A_\mu,\varphi)}e^{-\mathcal B}
%   \right]_{\msbar,~\mu=1/R}.
% \end{align}
% Each contribution is approximated by the following.

\begin{itemize}
 \item Higgs

       \begin{align}
        -\left[\ln\mathcal A'^{(h)}\right]_{\msbar} = -0.99192944327027 + 2.5\ln|\lambda| -
        3\ln\mu R.
       \end{align}
 \item Scalar

       Let $x = \kappa/|\lambda|$.
       For $x<0.7$,
       \begin{align}
        -\left[
         \ln\mathcal A^{(\sigma)}
        \right]_{\msbar}
         & = -0.239133939224974 x^2 + 0.222222222222222 x^3\nonumber                 \\
         & \hspace{3ex} - 0.134704602106396 x^4 + 0.102278606592866 x^5\nonumber     \\
         & \hspace{3ex} - 0.0839329261179402 x^6 + 0.0715956882048009 x^7\nonumber   \\
         & \hspace{3ex} - 0.0625481711576628 x^8 + 0.0555697470602515 x^9\nonumber   \\
         & \hspace{3ex} - 0.0500042455037409 x^{10} - 0.333333333333333 x^2\ln\mu R.
       \end{align}
       For $x>0.7$,
       \begin{align}
        -\left[\ln\mathcal A^{(\sigma)}\right]_{\msbar}
         & = -0.0261559272783723 + 0.0000886704923163256/x^4\nonumber                   \\
         & \hspace{3ex} + 0.0000962000962000962/x^3 + 0.000198412698412698/x^2\nonumber \\
         & \hspace{3ex} + 0.00105820105820106/x + 0.111111111111111 x\nonumber          \\
         & \hspace{3ex} - 0.181204187497805 x^2 + (-0.0055555555555556\nonumber         \\
         & \hspace{3ex} + 0.166666666666667 x^2)\ln x - 0.333333333333333 x^2 \ln\mu R.
       \end{align}
 \item Fermion

       Let $x=y^2/|\lambda|$.
       For $x<1.3$,
       \begin{align}
        -\left[\ln\mathcal A^{(\psi)}\right]_{\msbar}
         & = 0.64493454511661 x + 0.005114971505109 x^2\nonumber                                  \\
         & \hspace{3ex} - 0.0366953662258276 x^3 +  0.00476307962690785 x^4\nonumber              \\
         & \hspace{3ex} - 0.000845451274112082 x^5 + 0.000168244913551417 x^6\nonumber            \\
         & \hspace{3ex} - 0.0000353785958610453 x^7 +  7.67709260595572\times10^{-6} x^8\nonumber \\
         & \hspace{3ex} + (0.66666666666667 x +  0.333333333333333 x^2) \ln\mu R.
       \end{align}
       For $x>1.3$,
       \begin{align}
        -\left[\ln\mathcal A^{(\psi)}\right]_{\msbar}
         & = -0.227732960077634 + 0.00260942760942761/x^3\nonumber                  \\
         & \hspace{3ex} +  0.00271164021164021/x^2 + 0.00820105820105820/x\nonumber \\
         & \hspace{3ex} + 0.53790187962670 x +  0.296728717591129 x^2\nonumber      \\
         & \hspace{3ex} + (-0.06111111111111111 - 0.3333333333333333 x\nonumber     \\
         & \hspace{3ex} - 0.1666666666666666 x^2) \ln x\nonumber                    \\
         & \hspace{3ex} + (0.66666666666667 x + 0.333333333333333 x^2) \ln\mu R.
       \end{align}
 \item Gauge

       Let $x=g^2/|\lambda|$.
       For $x<1.4$,
       \begin{align}
        -\left[\ln\mathcal A'^{(A_\mu,\varphi)}\right]_{\msbar} = & 
         -0.96686103284373 - 1.76813696868318 x\nonumber                          \\
         & + 0.28259818232508 x^2 + 0.145084271024101 x^3\nonumber       \\
         & - 0.0241469799983579 x^4 + 0.00555917805602827 x^5\nonumber   \\
         & - 0.00145020891759152 x^6 + 0.000402580447036276 x^7\nonumber \\
         & - 0.000115821925959136 x^8 \nonumber         \\
         & + 0.5\ln|\lambda| + (-0.333333333333333 -  2 x -  x^2) \ln\mu R -\ln J_G.
       \end{align}
       For $x>1.4$,
       \begin{align}
        -\left[\ln\mathcal A'^{(A_\mu,\varphi)}\right]_{\msbar}
         & = -27.0091748854198 + 0.000266011476948977/x^4\nonumber                               \\
         & + 0.000288600288600289/x^3 + 0.000595238095238095/x^2\nonumber           \\
         & + 0.00317460317460317/x + 1.56519636465016 x\nonumber                    \\
         & - 0.41321696358277 x^2 + (-3.54033527491510\times 10^{-6}/x^{5}\nonumber \\
         & - 0.0000404609745704583/x^{4} - 0.00051790047450187/x^{3}\nonumber       \\
         & - 0.0082864075920299/x^{2} - 0.265165042944955/x\nonumber                \\
         & + 4.24264068711929) \sqrt{x}\arcsin
        \left[
         \frac{s}{\sqrt{s^2+79164837199872 x^9}}
        \right]\nonumber                                                                         \\
         & + (-6.01666666666667 +  0.5 x^2)\ln x\nonumber                           \\
         & + 1.5\ln[3.14159265358979 (-98796.7402597403\nonumber                    \\
         & + 136316.571428571 x -136594.285714286 x^2\nonumber                      \\
         & + 92160 x^3 +7372800 x^4 + 6553600 x^5)]\nonumber                        \\
         & +  0.5 \ln|\lambda| + (-0.333333333333333 -  2 x - x^2) \ln\mu R -\ln J_G,
       \end{align}
       where
       \begin{align}
        s = 7 + 80 x +1024 x^2 + 16384 x^3 +524288 x^4 -  8388608 x^5.
       \end{align}
\end{itemize}

%\bibliographystyle{apsrev}
\bibliographystyle{JHEP}
\bibliography{extramatter}

\end{document}
